\section{Latvia}

\subsection{Introduction}

After the fall of the USSR, the new republics moved quickly to legitimate their
national languages and put them to use in all areas of public life. As young
nation-states, it was considered imperative to assert a self-sufficient
linguistic identity after years under the influence of Russian. Moveover, the
threat of Global English was salient, with the US standing alone as the world's
superpower after the Cold War.

Latvia's constitution establishes Latvian as the official language
\parencite{lat-const}. A 1999 law and associated regulations apply the law to
names, requiring that non-Latvian names be transcribed to conform with Latvian
spelling and grammar on official documents \parencite{lat-lang}. This policy
was challenged under international human rights law on two occasions: in the
European Court of Human Rights in \textit{Mentzen v. Latvia}
\parencite{mentzen04} and under the International Covenant of Civil and
Political Rights in \textit{Raihman v. Latvia} \parencite{raihman10}. Although
the two tribunals reached different verdicts, the reasoning employed by the
contending parties and by the courts reveals insight into the role names play
in the language and in linguistic policy. The Latvian cases indicate that the
official ideology in Latvia embraces a conservative and literary variety of the
language as a national heritage and that the promotion of an official state
language is regarded as a legitimate function of the modern state.

\subsection{Language law in Latvia}

The Latvian Constitution, Chapter I, paragraph 4 states that "the Latvian
language is the official language of the Republic of Latvia"
\parencite{lat-const}. This clause is specifically implemented by the Official
Language Law of 1999. The language law contains numerous provisions that
outline the use of Latvian in all areas of modern life. With respect to
personal names, section 19 says that "names of persons shall be presented in
accordance with the traditions of the Latvian language and written in
accordance with the existing norms of the literary language"
\parencite{lat-lang}. This means that the authorities will Latvianise foreign
names when included on official documentations. Latvianisation includes two
processes, the addition of appropriate case endings (which depend on gender)
and the phonetic transcription of names into Latvian script.

We should note that the Latvian process of transcribing even names already
written in the Latin alphabet differs from the literal reproduction used by
most countries in this situation \parencite{varennes15}. To this end,
regulation 96, 2002 on "the transcription and use of names of foreign origin in
the Latvian language" sets out in impressive detail how names should be
transcribed, including sections for each common foreign language in Latvia with
specific examples \parencite{notei96}. For instance, converting the name "John
Wright" into Latvian produces "Džons Raits", which retains the same phonetic
value as the English original and adds the obligatory masculine nominative
suffix "-s". Because of Latvian declension rules, males and females with the
same surname have different endings attached. According to the regulation,
President Hoover would be known as "Hūvers", while his First Lady would go by
"Hūvere". In the case of transcribed names, an earlier resolution provides that
the original form of the name may appear on the passport with a certified
stamp, if the person desires. The document states that "the form of the surname
written in Latvian shall be legally identical to the original form of the
surname, the historical form, or the form transliterated into Latin
characters" \parencite{notei295}.

\subsection{\textit{Mentzen v. Latvia}}

Juta Mentzen, a Latvian woman, married her husband in Bonn in 1998 and took his
German surname "Mentzen". In 1999, she requested a new passport from the
Latvian Interior Ministry, specifically asking that her married name be
transcribed literally. The request was ignored, however, and she received a
passport with the surname listed as "Mencena", reflecting the Latvian use of
`c' to denote [ts] and the feminine nominative suffix "-a". Pursuant to the
regulations, the original form was certified as "Mentzen" on a separate page of
the passport. Mentzen found little success appealing within the Latvian court
system for relief, although the Constitutional Court ruling against her did
specify that the "original form" should be moved from page 14 to page 4 of the
passport to be closer to the main page.

Apparently this did not satisfy Mentzen, who filed a complaint with the
European Court of Human Rights, claiming that Latvia had violated her right to
private and family life guaranteed by Article 8 of the European Convention on
Human Rights. This article prohibits interference by authorities with this
right except in accordance with the law and as necessary for promoting the
legitimate interests of the state \parencite{echr}. It was admitted by all
parties that an Article 8 interference had occurred and that such interference
was lawful, since it was set out clearly in legislation and applied to all
names equally. The legal substance of the case was to determine whether
promotion of the national language by regulating names is a "legitimate" state
function.

The Latvian government's defence of the law rested on the need to preserve the
structural integrity of Latvian and the desirability of protecting the language
after years of Soviet rule. They state in their submission "that Latvian had
only been able to survive as a result of the Latvian people's determination to
carry on using and, to the extent possible, to promote the use of their
language" \parencite{mentzen04}. But official Latvian ideology does not simply
call for the use of Latvian in any form. It specifically promotes the
conservative literary form, noting that the language laws "aimed at protecting
the right of others to hear and use \textit{correct}\footnote{my emphasis}
Latvian on Latvian territory". Even the practice of transcribing foreign names,
Latvia claims, is an ancient Latvian tradition, dating back to the first text
published in Latvian, \textit{Parvus catechismus catholicorum} by Saint
Cainisius in 1585. They also claim that allowing literal transcription would
threaten the case system's integrity and threaten to degrade the language's
intelligibility: "Systematically permitting names to be entered in passports
without endings would encourage people to use the same form in conversation.
Once such usage had become commonplace, it would open the door to the
deformation of the language and its deterioration on a vast scale."

Mentzen countered on a number of linguistic grounds. As a living language,
"Latvian could not be isolated from its environment" and "was inevitably
influenced by other languages". Her reply notes that foreign persons living in
Latvia, such as her husband, tended not to have their names transcribed.
Moreover, she notes that foreign words and institutional names commonly appear
in Latvian media, while trademarks including foreign words or names are
permitted in non-Latvianised form, even if they are sometimes placed in
italics. As Latvian already has indeclinable nouns, the concern raised by the
government regarding a communication breakdown is apparently unfounded, since
speakers can negotiate semantic roles based on context. She claims that she had
personally experienced difficulties from having a different surname from her
husband, complicating private transactions and travel abroad and undermining
her choice to be permanently identified with her husband's family.

The Court decided in favour of the government, ruling that promoting correct
Latvian usage was a legitimate state function. They set out their view of an
official language as follows: "A language is not in any sense an abstract
value. It cannot be divorced from the way it is actually used by its speakers.
Consequently, by making a language its official language, the State undertakes
in principle to guarantee its citizens the right to use that language both to
impart and to receive information \dots implicit in the notion of an official
language is the existence of certain subjective rights for the speakers of that
language."

Although the Court acknowledged interference with Mentzen's privacy, they
determined that the measure was necessary, legitimate, and adequately tempered
by Latvia's willingness to place the original form of the name on the passport
nearby the Latvianised form. Thus the ruling claims that names can legitimately
be used to promote linguistic behaviours in public life.

\subsection{\textit{Raihman v. Latvia}}

The United Nations Human Rights Committee considered essentially the
same question as the European court, this time under the International Covenant
on Civil and Political Rights. The case was brought by Leonid Raihman, a
Latvian national and member of the Russian-Jewish minority. Raihman used his
name in the original form from the time of his birth in the USSR until 1998,
when Latvian officials granted him a new passport with his name Latvianised as
"Leonīds Raihmans", adding the masculine nominative suffix \textit{-s}. He
achieved even less success in Latvian courts than Mentzen, as he filed the
complaint after her case had already established precedent in the
Constitutional Court. Taking his complaint to the international level, he
claimed that the Latvian policy interfered with his rights under Article 17 of
the Covenant, which states that "no one shall be subjected to arbitrary or
unlawful interference with his privacy, family, home or correspondence"
\parencite{iccpr}. Latvia reiterated its defence from Mentzen that the policy
was lawful and appropriate to the goal of protecting Latvian usage.

The UN's decision differed with the ECHR's in finding that the policy
arbitrarily interfered with Raihman's privacy under Article 17, stating that
"the Committee considers that the forceful addition of a declinable ending to a
surname, which has been used in its original form for decades, and which
modifies its phonic pronunciation, is an intrusive measure, which is not
proportionate to the aim of protecting the official State language"
\parencite[8.3]{raihman10}.

\subsection{Remarks}

We should note that the two courts agree on most points of these very similar
cases. They concur that states may legitimately promote an official language
and that the interference in privacy presented by the Latvian policy was
lawful, stemming from legislation that sets out general principles leaving
little room for individual discretion on the part of administrators. Moreover,
both the ECHR and UNHRC admit that Latvia's history of struggle under the
Soviet régime and its strict case system are particularities that must be taken
into account for ensuring the survival of Latvian.

However, ECHR determined that individuals' names were an appropriate forum for
pursuing public linguistic policy, while UNHRC felt that the measure was
unjustified. The UNHRC's favourable ruling for the petitioner may be influenced
by the Committee's prior establishment of a "right to choose and change one's
own name" under Article 17 of the Covenant, which they easily extended to
protect names from undergoing "passive" changes by the State as well.
Ultimately, these two cases reveal that the line between acceptable and
unacceptable regulations on naming is constantly shifting and not yet clearly
defined, even by human rights scholars.
