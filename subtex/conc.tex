\section{Conclusion}

This paper presents four case studies on naming restrictions which highlight
the extent to which naming law aligns with official language policy and how
names are considered a venue for enacting linguistic change. Although naming
restrictions are common in today's bureaucratic age, we consider the historical
context of naming to underscore the importance of protecting individual
autonomy in naming. Although no name should be an island---inaccessible for
public use---it remains ultimately the domain of the individual, as an
expression of personal identity.

The case studies chosen herein suggest two varieties of naming restrictions
with respect to language policy. The cases from the United States and China are
passive restrictions, deriving primarily from the digital infrastructure chosen
by the government. Although these systems ultimately support the language
policy (Proposition 63 for California and Pǔtōnghuà for China), this primarily
reflects the state infrastructure that has been built to homogenise the
linguistic landscape, rather than a premeditated curating of names for
linguistic value. Neither Chinese nor English are in serious danger by the
presence of nonstandard names.

For the small nations in our sample, Iceland and Latvia, we do see a
self-conscious articulation of linguistic policy and a perception of the threat
to their national languages. Both governments see the form of names as part of
the language and necessary to maintain in their traditional form. Small
languages face a real threat of obsolescence and marginalisation as "heritage"
languages, having their active role in public life circumscribed by a larger,
especially colonial, language. Both countries claim a need for linguistic
purism, protecting the traditional structure of the language. This serves as
their rationale for preventing onomastic innovation. 

Despite the legitimate need to protect small national languages from the
onslaught of globalisation, this paper notes many legal considerations of
naming interference under both human rights and American law. Both traditions
prioritise the individual's right to choose names for themselves and their
children. Any proposed abridgement of this right should be carefully reviewed
to ensure that the measure is necessary and effective for the aim sought. I
hope that this paper establishes the inherent connection between naming law and
language policy, although further work is needed to fully illuminate the
relationship, apply it to a wider range of jurisdictions, and classify the
world's many naming laws according to their relationship with language policy.
