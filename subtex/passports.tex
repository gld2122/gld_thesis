\section{Passports}

The standard protocol governing machine-readable travel documents (MRTD) is
Document 9309 issued by the International Civil Aviation Organization
\parencite{ICAO9309}. Since all countries must operate on a shared standard, the
diplomatic community has forged a compromise between cultural diversity and
international security. The 9309 standard provides sufficient flexibility to
adapt to the wide variety of languages and scripts employed worldwide.

A 9309-compliant MRTD data page is divided into two sections: the Visual
Inspection Zone (VIZ) and the Machine Readable Zone (MRZ).

\subsection{Visual Inspection Zone}

The top third of the data page is intended for the review of border officials at
the time of entry. Countries may populate the required VIZ fields as desired in
the state's national language, provided a transcription or translation is
provided into English, Spanish, or French. Thus, modernized passports do not
\textit{per se} coax a country toward the adoption of standard alphabets;
however, they do ensure effcient intercommunication in the world's
\textit{scripta franca} the Latin alphabet.

\subsection{Machine Readable Zone}

In contrast, the content of the Machine Readable Zone is highly controlled. The
only characters allowed in the two lines of the MRZ are those belonging to a
defined ASCII subset: (0-9, A-Z, and <). Moreover, these characters must be
printed in the typeface OCR-B (OCR=Optical Character Recognition) using
character and line spacings strictly defined in the 9309 standard. The adherence
to these guidelines allows for unambiguous machine recognition.

\includegraphics{subtex/9309.4.3.2.4.png}

For Latin alphabet languages, most characters containing diacritics simply have
the mark dropped, although some characters have recommended encodings to
losslessly transliterate the character. The document spells out a more extensive
scheme for Cyrilic and Arabic characters which allows nearly lossless recovery
of the original form from the highly schematic MRZ specification. There is even
a sample Python program for converting from the MRZ to Unicode Arabic.

\subsubsection{Latin}

The ICAO tries to account for the varying importance of diacritical marks in
Latin-based scripts. Those such as the acute or grave accents, which appear over
vowels mainly for the purpose of clarifying pronunciation, are simply eliminated
in the MRZ. However, other characters receive recommended encoding methods These
are the more "salient" diacritic characters, such as the German umlauts (ä, ö,
ü) or the Spanish ñ, which in their respective languages are considered separate
letters, rather than a variation on the unaccented form. The following table
shows the special encodings recommended for European diacritics; all other
characters simply have the mark dropped:

\begin{tabular}{l|l|l|l}
\textbf{Unicode} & \textbf{National Character} & \textbf{Description} &
\textbf{Recommended transliteration} \\
00C4 & Ä & A diaeresis & AE or A \\
00C5 & Å & A ring above & AA or A \\
00C6 & Æ & ligature AE & AE \\
00D1 & Ñ & N tilde & N or NXX \\
00D6 & Ö & O diaeresis & OE or O \\
00D8 & Ø & O stroke & OE \\
00DC & Ü & U diaeresis & UE or UXX or U \\
00DE & Þ & Thorn (Iceland) & TH \\
00DF & ß & double S (Germany) & SS \\
0132 & IJ & ligature IJ (Netherlands) & IJ \\
0152 & Π& ligature OE & OE \\
\end{tabular} (\parencite{ICAO9303} 3.6.A)

The name "Térèsa Cañón" would become CANXXON<<TERESA in the MRZ. The ñ is
encoded in the MRZ, while no distinction is made of the é or è. Likewise, the
German name "Wilhelm Furtwängler" would become FURTWAENGLER<<WILHELM (ä becomes
AE). (b.4.2) Although it leaves a large set of European characters
unrepresented, it would not be difficult to expand the escape sequence system to
represent additional diacritical marks. (An interesting edge case would be a
Spanish traveller named José Nuñenxx.)

\subsubsection{Cyrillic}

The ICAO transcription system for Cyrillic characters permits a nearly
one-to-one transliteration between the MRZ and the name in the original
language. The system recognizes the different values that a Cyrillic glyph might
take in various languages. For example the letter Ю is transliterated as "IU",
unless it is the first character of a Ukrainian name, in which case "YU" is
permitted. Likewise for Щ; this is SHCH, except in Bulgarian, where it is SHT.

\subsubsection{Arabic}

For example, the Arabic name {\arfont ابو بكر محمد بن زكريا الرازي} would be
rendered in the MRZ as ABW<BKR<MXHMD<<BN<ZKRYA<ALRAZY.

Although the name looks almost incomprehensible to a human, the weird encoding
actually allows the one-to-one mapping between the MRZ and the original Arabic
name. See more examples in the figure below:

\includegraphics{subtex/9309.3-appendix-b.5.9.png}
