\section{Functions of Names}

Personal names have three main functions, which \textcite[392]{heymann11} calls
\textit{denotative}, \textit{connotative}, and \textit{associative}. The
canonical use of a name is for denotation or reference, serving as a linguistic
token selecting an individual for discussion or address. Connotative and
associative information is less direct but no less present in a name, as it can
convey cultural information about the bearer such as nationality, ethnicity,
religion, etc. In some cases, the two functions are opposed to one another, and
a balance must be struck between legibility and free expression.

Name-giving is a cultural universal. Ethnographers have found no society past
or present that does not name individuals \parencite{alford87}. So naming is
deeply rooted as an aspect of human identity. Historians and ethnographers tell
us that preindustrial naming differs greatly from industrialised practices.
Nuers of Sudan take two given names (maternal and paternal), a ceremonial clan
name, and an ox name of a favourite domestic animal \parencite{wardhaugh92}.
Likewise, nicknames are liberally bestowed based on traits and accomplishments
and used widely by in-group members in groups like the Giriama of Kenya
\parencite{parkin88} \parencite{wardhaugh92}. The idea of \textit{fixing} the
name for legal purposes is not a given. Names can vary by context and across
time and space.

With bureaucratisation and industrialisation, we see also the canonicalisation
of name. A mobile population and impersonal state presence requires accurate
and speedy identification of individuals. Hence \textcite{scott02} observe that
patrilineal surnames tend to arise wherever centralisation is occurring: Qín
China, Norman England, Atatürkian Turkey. As they suggest, patrilineal surnames
are perfectly suited to state functions requiring clear identification of
individuals: taxation, conscription, and justice \parencite[18]{scott02}. In
fact, the ideal name from an administrative point-of-view is not a name at all,
but a unique serial number assigned to every person.\footnote{inuit
\parencite{scott02}}

The point of standardising names is to produce a particular style of knowledge
about the populace, which is legible, objective, and public. So the name
becomes a tracker to follow the person's movements, marriages, kinship, and
careers, with the emphasis being placed on the individual's relation to society
rather than the individual themselves. This is not to deny the real importance
of administrative factors in the modern role of names. Names, as the primary
way of referring to other individuals, have a public function of address and as
such must be legible and usable for most people. But we should be mindful of
the individual's traditional autonomy in naming and ensure that any
abridgements thereof by modern bureaucracies are limited and truly necessary
for the public interest.

\subsubsection{Symbolic Functions}

\begin{aquote}{Janice "Lokelani" Keihanaikukauakahihuliheekahaunaele
	\parencite{lee-valley}}
	You see, to some people in the world, your name is everything. If I say my
	name to an elder Hawaiian (kupuna), they know everything about my husband's
	family going back many generations…just from the name.
\end{aquote}

The symbolic function of a name is the complex of information that conveys a
complex of information, be it genealogical, cultural, linguistic, religious,
etc. The name is a foundational aspect of a person's identity and an important
choice that parents make in child-raising. Parents' motives in choosing a name
vary widely, but in general, child's names serve either to express some special
aspirations that parents have for the child or to honour a particular family
relationship, like a grandfather or aunt \parencite{finch08}. These two aspects
must be balanced, and they sit on a spectrum between what \textcite{finch08}
calls individualizing functions and connecting functions.

The individualizing functions referred to by \textcite{finch08} are those
aspects of a name which make a statement on the individual themselves,
including the associative and connotative. The clearest manifestation of this
is the choice of a child's forename, although this choice clearly makes more of
a statement on the parents than the child itself: "In selecting a name
(especially for a first-born child) parents are not only determining the
personhood of their child but are also taking a key step in defining their own
new identity as parents." \parencite[718]{finch08} (cf.
\parencite[399]{heymann11}) Hence a parent can name their child something
"beautiful" like "Isabella" or something "strong" like "Samson", imparting the
connotations of these names without any knowledge of the newborn's future
personality.

The related function of association entails a name that is the same as or
similar to that of a person whose memory would be honoured or evoked by use of
the name. In the case above, "Samson" associates the name's bearer with the
biblical hero. A name like "Adolf Hitler Campbell" associates one to an
infamous historical figure. 

Connecting functions are facets that locate the individual within larger
milieu.  This takes the form of surnames, which in Anglophone societies
identify the paternal family unit to which the individual belongs. "The
construction of a name, and its uses through a lifetime, also can embody a
sense of connectedness with family - with the parents who gave the name, and
with others in a domestic arrangement or a kin network with whom all or part of
the name is shared." \parencite[711]{finch08} Other connecting functions
include giving a first name after an older ancestor connects you to a more
specific family relationship or even the linguistic or religious connotations
carried within the first or last name, which can connect a person to or set
them apart from the dominant society in which they live.

\subsection{Legal Perspectives on Naming}

The English common law tradition is relatively lax on the form of names given
at birth or adopted by adults, as it recognises the right of individuals to
adopt any name they choose, unless it used for a fraudulent purpose
\parencite[403]{heymann11}. Moreover, the designation of a person's name need
not be effected by a state or court official, although they may seek such an
order for its legitimizing values. Rather a name becomes a name through use of
the name, in that common law considers personal names primarily for their
denotative function, and courts will look for examples of the name used in
context to establish that it indeed carries the denotative function. For
example, NY Superior Court rejected a name change from "Anatoly Eyzenberg" to
"Tony Eisenberg" for a City Council candidate who hoped to appear more American
and less Russian. The Court's reason for rejecting rested on the fact that the
petitioner had not actually used the "new name" on any of his personal
documents. \parencite{eisenberg03} \parencite[404]{heymann11}.

Courts have a public interest in maintaining the relative usefulness and
legibility of names, although we note that \textit{absolute} unambiguity is
neither possible nor desirable, if we consider common names like "John Smith".
Despite the hands-off approach of common law on names, American courts tend to
intervene when the denotative aspect of a name is threatened by its inherent
content, for example the oft-cited case of a person named "Adolf Hitler". This
name can certainly function for reference, but for names which are explicitly
offensive, many courts are reluctant to apply state "imprimatur" on the name
change \parencite[413]{heymann11} and moreover because "the connotations
associated with such names are so strong that they frustrate the ability of the
names to server their denotative function" \parencite[417]{heymann11}. For
example, an Ohio court denied an application for name change to "Santa Robert
Claus" because of the people's "proprietary right in the identity of Santa
Claus" \parencite[419]{heymann11} \parencite{handley00} and a Pennsylvania
court denied "World Savior", noting that "no one has ever asked us to invoke
our legal authority to certify them as a savior of man"
\parencite[419]{heymann11} \parencite{bethea91}.

Civil law jurisdictions a
