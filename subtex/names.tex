\section{Names}

\begin{aquote}{Janice "Lokelani" Keihanaikukauakahihuliheekahaunaele}
You see, to some people in the world, your name is everything. If I say my name
to an elder Hawaiian (kupuna), they know everything about my husband's family
going back many generations...just from the name. \parencite{lee-valley}
\end{aquote}

Nearly all human beings have one or more names. Article Seven of the Convention on
the Rights of the Child enshrines a person's fundamental right to a name: "The
child shall be registered immediately after birth and shall have the right from
birth to a name." \parencite{crc}

\subsection{A Survey of the World's Names}

Besides the fact that names are given in all languages and scripts of the world,
the form and significance of names varies immensely. Anglo-American practice
entails two given names, the "first" and "middle" names, appended to a
patrilineal surname. In many Hispanophone cultures, a child receives both a
patrilineal and a matrilineal surname, with the father's surname taking
precedence in terms of identification. An Icelandic surname consists of the
father's given name with the attached suffix -son or -dottir, depending on
gender. A similar practice occurs in Pakistan, without any suffixation on the
father's given name. \textcite{finch08} reports that many South Indian persons
have three names, a personal name, a family name, and a village name.

The ordering of a name's elements also varies. East Asian and Hungarian names
reverse the Western order, putting the given name after the family name.
Standard Chinese given names consist of two characters, whose meanings may or
may not be interconnected (Emma Woo). \textcite{wardhaugh92} cites data from
\textcite{evans-pritchard48} regarding Sudan's Nuer people: Nuers receive both a
paternal and maternal given name, a ceremonial clan name, and take for
themselves an "ox name" from a favorite domestic animal.

From a high-level survey it is clear that great diversity exists around names
worldwide; but we should not view each one as a static system. Instead, a naming
culture constitutes a space of social rules and expectations which allow for
cultural expression thru individual acts. On one end, a name can be used to
locate the individual within a subgroup of the society, due to the name's
linguistic, cultural, or religious connotations. For example, religion has been
an important influence on naming; bearing a Christian or Islamic name marks
someone as a likely member of that religious group. The linguistic origin of a
name may also convey information. People often ask the origin of my last name,
DeFreitas, which comes from Portuguese, a language I do not speak or have any
ancestral connection to. In other cases, a name's origin may alert you
successfully to the bearer's ancestry. In Kenya, we find the Giriama group,
whose clan name system identifies not only the bearer's clan, but also provides
information about the bearer' generation and birth order within the clan
\textcite{parkin89}.

\subsection{Functions of a Name}

Why do people have names? We can first discern a top-level division in the
functions of names between \textit{reference} and \textit{symbolic} functions,
which categories we will further subdivide below.

\subsubsection{Reference Functions}

Consider a table that contains one row for every person with whom you are
acquainted. We store everything we know about a person in their database entry:
nationality, appearance, favorite foods, pet peeves, etc. But faced with this
vast set of data, we need each row (person) to have one relatively unique
identifier with which we can access their entry and retrieve the other fields.
(Using hair color, for example, would be a poor choice. The datum "brown hair"
does not narrow our search to a single individual.) A personal name is this
token by which we single out a specific individual, a concise token which makes
reference to the larger concept of a given human being.

The reference function of a name does not depend on the name's content; any
unique or nearly unique token allows us to navigate the database. It is not
clear that "John" is a more effective identifier than "12345678"; indeed the
latter seems more likely to be unambiguous.

\subsubsection{Symbolic Functions}

The symbolic function of a name is played by the name's content itself; the
complex of information that the linguistic form conveys, be it genealogical,
cultural, linguistic, religious, etc. This is the function that 

We can further subdivide a name's symbolic functions; they sit on a spectrum
between what \textcite{finch08} calls individualizing functions and connecting
functions.

Individualizing functions are those aspects of a name which make a statement on
the individual themselves. The clearest manifestation of this is the choice of a
child's forename; indeed this makes more of a statment on the parent than the
child itself. "In selecting a name (especially for a first-born child) parents
are not only determining the personhood of their child but are also taking a key
step in defining their own new identity as parents." \parencite[718]{finch08}
Hence a parent can name their child something "beautiful" like "Isabella" or
something "strong" like "Samson".

Connecting functions are facets that locate the individual within a larger
milieu. Most basically this takes the form of surnames, which in Anglophone
societies identify the paternal family unit to which the individual belongs.
"The construction of a name, and its uses through a lifetime, also can embody a
sense of connectedness with family - with the parents who gave the name, and
with others in a domestic arrangement or a kin network with whom all or part of
the name is shared." \parencite[711]{finch08} We can find more subtle connecting
functions, however. Choosing a first name after an older ancestor connects you
to a more specific family relationship. And even the linguistic or religious
connotations carried within the first or last name can connect a person to or
set them apart from the dominant society in which they live.

\subsection{Digital Names}

This paper explores a modern tension between the name's reference and symbolic
function: computers. As the world has become computerized and the world's
information stored in databases, At the same time, a name cannot remain the sole
property of its owner; it must facilitate interaction with the wider world as a
means of address. If a name affirms your status as an individual, it no less
affirms your status as a citizen of your country, resident of your city,
customer of your electric service provider, holder of your credit card, employee
of your company, and recipient of your parking ticket. A name is worth nothing
if others people in the environs cannot pronounce it, write it, or remember it.

\subsection{Computers}

In the "digital age" has changed writing from a free and individualized practice
to a one based in a discrete and logical structure consisting of discrete
glyphs. Since most early development of computers took place in the United
States, English gained a natural ascendancy over other languages in the field of
digital communication. English, perhaps as a coincidence, is also one of the
easiest languages to represent in code, requiring at the bare minimum just the
26 non-accented characters of the English alphabet, perhaps with some
punctuation and numbers. The ASCII standard encoding, with 127 available code
points, is more than enough to represent the English language in digital form.
Thus, organizations which deal only infrequently with non-English text have been
slow to update their databases to Unicode standards.
