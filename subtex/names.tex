\section{Names}

\begin{aquote}{Janice "Lokelani" Keihanaikukauakahihuliheekahaunaele
	\parencite{lee-valley}}
	You see, to some people in the world, your name is everything. If I say my
	name to an elder Hawaiian (kupuna), they know everything about my husband's
	family going back many generations…just from the name.
\end{aquote}

Personal names convey religious, cultural, and linguistic information about
their bearer. The semantic relation between a name and the bearer is hardly
one-to-one.

Nearly all human beings have at least one name. Naming is considered a cultural
universal by ethnographers; no society has been found which does not assign
names for reference to specific individuals \parencite{alford88}. Several human
rights treaties make reference to a person's right to a name. Article Seven of
the Convention on the Rights of the Child even codifies a person's fundamental
right to a name: "The child shall be registered immediately after birth and
shall have the right from birth to a name." \parencite{crc} The Convention does
not comment, however, on the right of parents to a name \textit{of their
choice}. This is because of lagging recognition of the value of a name for
conveying cultural value.

\subsection{Functions of a Name}

Why do we give people names? Let's divide naming functions into
\textit{reference} and \textit{symbolic} functions, categories which we will
subdivide further below.

\subsubsection{Reference Function}

Consider a table containing one row for every person with whom you are
acquainted. You store everything you know about the person in their database
row: nationality, appearance, favorite foods, pet peeves, etc. But faced with
this vast set of data, each row (person) needs to have one relatively unique
identifier (key in database terminology) by which we can access their entry and
retrieve the other fields. (Using hair color, for example, would be a poor
choice. "Brown hair" does not narrow our search to a single individual.) A
personal name is such a token by which we single out a specific individual, a
making reference to the larger concept of a given human being.

The reference function of a name does not depend on the name's content; any
unique or nearly unique token allows us to navigate the database. It is not
clear that "John" is a more effective identifier than "12345678"; indeed the
latter seems more likely to be unambiguous. As \textcite{scott02} notes, "serial
numbers" are an administrator's dream. Of course, they are too obscure for daily
use, but institutions have worked nonetheless to make personal names more
legible. \textcite{scott02} argue that this need for "legibility" a primary
explanation for the standardized use of patrilineal family surnames by modern
states. For example, modern surnames were adopted by Norman élites following the
conquest in order to emphasize their property holdings. Elsewhere, in Turkey,
modern surnames were only adopted in 1934 under the modernising régime of
Atatürk \parencite{scott02}.

\begin{aquote}{\parencite{scott02}}
The rise of the permanent patronym is inextricably associated with those aspects
of state-making in which it was desirable to distinguish individual (male)
subjects: tax collection (including tithes), conscription, land revenue, court
judgements, witness records, and police work.
\end{aquote}

Thus we can see digital recordkeeping of persons and places as the next step in
the streamlining of "social information". Institutions are able to organise and
manipulate data about people with unprecedented efficiency, but this requires
standardisation of the data to make it consistent. But the more people's names
are shoehorned into a database-friendly format, there is vast social information
and expressive content being lost, especially if someone's name must be altered
simply to be stored in the digital format.

\subsubsection{Symbolic Functions}

The symbolic function of a name is that in which the name's content itself
conveys a complex of information, be it genealogical, cultural, linguistic,
religious, etc. This is the function that digital records threaten to eliminate,
as it is fluid and ephemeral, whereas computers require complete unambiguity.

We can subdivide a name's symbolic functions into two aspects that must be
balanced; they sit on a spectrum between what \textcite{finch08} calls
individualizing functions and connecting functions.

Individualizing functions are those aspects of a name which make a statement on
the individual themselves. The clearest manifestation of this is the choice of a
child's forename; indeed this makes more of a statment on the parent than the
child itself. "In selecting a name (especially for a first-born child) parents
are not only determining the personhood of their child but are also taking a key
step in defining their own new identity as parents." \parencite[718]{finch08}
Hence a parent can name their child something "beautiful" like "Isabella" or
something "strong" like "Samson".

\begin{aquote}{Albus Dumbledore}
"Call him Voldemort, Harry. Always use the proper name for things. Fear of a
name increases fear of the thing itself." \parencite{rowling97}
\end{aquote}

Connecting functions are facets that locate the individual within larger milieu.
this takes the form of surnames, which in Anglophone societies identify the
paternal family unit to which the individual belongs. "The construction of a
name, and its uses through a lifetime, also can embody a sense of connectedness
with family - with the parents who gave the name, and with others in a domestic
arrangement or a kin network with whom all or part of the name is shared."
\parencite[711]{finch08} We can find more subtle connecting functions, however.
Choosing a first name after an older ancestor connects you to a more specific
family relationship. And even the linguistic or religious connotations carried
within the first or last name can connect a person to or set them apart from the
dominant society in which they live.

\subsection{A Survey of the World's Names}

The form and contents of people's names vary immensely around the world.
Anglo-American practice entails two given names, the "first" and "middle" names,
being appended to a patrilineal surname. In many Hispanophone cultures, a child
receives both a patrilineal and a matrilineal surname, and the father's surname
taking precedence in terms of identification. An Icelandic surname consists of
the father's given name with the attached suffix -son or -dottir, depending on
gender. A similar practice occurs in Pakistan, but without any suffixation on
the father's given name. In South India many people have have three names, a
personal name, a family name, and a village name.

The ordering of a name's elements also varies. East Asian and Hungarian names
reverse the Western order, putting the given name after the family name.
Standard Chinese given names consist of two characters, whose meanings may or
may not be interconnected (Emma Woo). \textcite{wardhaugh92} cites data from
\textcite{evans-pritchard48} regarding Sudan's Nuer people: Nuers receive both a
paternal and maternal given name, a ceremonial clan name, and take for
themselves an "ox name" from a favorite domestic animal.

Even from a brief overview, it is clear that great diversity exists in names
worldwide; but we should not view each one as a static system. Instead, a naming
culture constitutes a space of social rules and expectations which allow for
cultural expression thru individual acts. On one end, a name can be used to
locate the individual within a subgroup of the society, due to the name's
linguistic, cultural, or religious connotations. For example, religion has been
an important influence on naming; bearing a Christian or Islamic name marks
someone as a likely member of that religious group. The linguistic origin of a
name may also convey information. People often ask the origin of my last name,
DeFreitas, which comes from Portuguese, a language I do not speak or have any
ancestral connection to. In other cases, a name's origin may alert you
successfully to the bearer's ancestry. In Kenya, we find the Giriama group,
whose clan name system identifies not only the bearer's clan, but also provides
information about the bearer' generation and birth order within the clan
\textcite{parkin89}.

Asking the cómo-se-llama in Western context presupposes a fixed answer. Imagine
the IRS' dismay if Walter demanded that he should addressed as Jean-Pierre for
the entirety of 2020. (It is, after all, a leap year.) The IRS would cease to
function if it had to honor such naming practices for 320 million citizens.
Administrative convenience, as \textcite{scott02} finds, has been the primary
factor promoting the spread of fixed legal names and hereditary surnames. For
example, in England, surnames were first adopted by the landed Norman élite. As
the bureaucracy strengthened and middling types aspired to emulate their status.
The practice spread down the social ladder and by the end of the 18th century
had reached all parts of Great Britain. An accelerated programme as such
occurred in Turkey starting in 1934 under the Westernising régime of Atatürk.

\subsection{Digital Names}

This paper explores a modern tension between the name's reference and symbolic
function: computers. As the world has become computerized and the world's
information stored in databases, a name cannot remain the sole
property of its owner; it must facilitate interaction with the wider world as a
means of address. If a name affirms your status as an individual, it no less
affirms your status as a citizen of your country, resident of your city,
customer of your electric service provider, holder of your credit card, employee
of your company, and recipient of your parking ticket. A name is worth nothing
if others people in the environs cannot pronounce it, write it, or remember it.

\subsection{Computers}

As we can see, fixed legal names are not a fact of life; they are a
construction. In many non-industrialised societies, we find that naming is fluid
and context-dependent. For example, the Giriama…

Fluid naming practices presuppose relative familiarity and the sharing of social
context. Such practices cannot withstand a growing bureaucratic presence;
governments need a "synoptic" view of their populations \parencite{scott02}.

Today, computers are making the synoptic view ever more crystal clear for
decision makers, providing precise and updated information from all parts of the
Empire. On top of the requirement of a fixed legal name, it is now expected (if
not mandated) that the name be compatible with institutional systems of digital
record-keeping. This is the name that will go on your birth certificate, your
passport, your driver's licence, your social security card; it is the name that
makes you You.

The "digital age" (to be fair, we should include the impact of typewriters as
well) has changed writing from an analogue and freeform practice to one based on
combinations of discrete and fixed glyphs. Since most early development of
computers took place in the United States, English gained a natural ascendancy
over other languages in the field of digital communication.  English, perhaps as
a coincidence, is also one of the easiest languages to represent in code,
requiring at the bare minimum just the 26 non-accented characters of the English
alphabet, perhaps with some punctuation and numbers.  The ASCII standard
encoding, with 127 available code points, is more than enough to represent the
English language in digital form.  Thus, organizations which deal only
infrequently with non-English text have been slow to update their databases to
Unicode standards.
