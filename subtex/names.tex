\section{Functions of Names}

The canonical job of a name is reference: a linguistic token selecting an individual for discussion or address. This role is complemented by a second: conveyance of cultural information about the bearer. In some cases, the two functions are opposed to one another, and a balance must be struck between legibility and free expression.

Name-giving is a cultural universal; ethnographers have found no society past or present that does not give names to individuals \parencite{alford87}. Naming is deeply rooted as a rite of human identity. We know from historians and ethnographers that preïndustrial naming differs greatly from industrialised practices. The Nuers of Sudan take two given names (maternal and paternal), a ceremonial clan name, and an ox name of a favourite domestic animal \parencite{wardhaugh92}. Likewise, nicknames are liberally bestowed based on traits and accomplishments (Giriama). The idea of \textit{fixing} the name is not a given; names can vary by context and across time and space.

Enter bureaucratisation and industrialisation, and we see concomitantly the canonicalisation of name. A mobile population or impersonal state presence requires accurate and speedy identification of individuals. Hence \textcite{scott02} observe that patrilineal surnames tend to arise wherever centralisation is occurring: Qin China, Norman England, Atatürkian Turkey. As they suggest, patrilineal surnames are perfectly suited to state functions requiring clear identification of individuals: taxation, conscription, and justice \parencite[18]{scott02}. In fact, the administrative dream is not a name at all, but a unique serial number assigned to every person.

Canonicalisation's raison d'être is producing a particular brand of knowledge, which is legible, objective, and public. So the name becomes a tracker to follow the person's movements, marriages, kinship, and careers, the emphasis being placed on the individual's relation to society rather than the individual themselves.

\subsubsection{Symbolic Functions}

\begin{aquote}{Janice "Lokelani" Keihanaikukauakahihuliheekahaunaele
	\parencite{lee-valley}}
	You see, to some people in the world, your name is everything. If I say my
	name to an elder Hawaiian (kupuna), they know everything about my husband's
	family going back many generations…just from the name.
\end{aquote}

The symbolic function of a name is that in which the name's content itself
conveys a complex of information, be it genealogical, cultural, linguistic,
religious, etc. This is the function that digital records threaten to eliminate,
as it is fluid and ephemeral, whereas computers require complete unambiguity.

We can subdivide a name's symbolic functions into two aspects that must be
balanced; they sit on a spectrum between what \textcite{finch08} calls
individualizing functions and connecting functions.

Individualizing functions are those aspects of a name which make a statement on
the individual themselves. The clearest manifestation of this is the choice of a
child's forename; indeed this makes more of a statement on the parent than the
child itself. "In selecting a name (especially for a first-born child) parents
are not only determining the personhood of their child but are also taking a key
step in defining their own new identity as parents." \parencite[718]{finch08}
Hence a parent can name their child something "beautiful" like "Isabella" or
something "strong" like "Samson".

\begin{aquote}{Albus Dumbledore}
"Call him Voldemort, Harry. Always use the proper name for things. Fear of a
name increases fear of the thing itself." \parencite{rowling97}
\end{aquote}

Connecting functions are facets that locate the individual within larger milieu.
this takes the form of surnames, which in Anglophone societies identify the
paternal family unit to which the individual belongs. "The construction of a
name, and its uses through a lifetime, also can embody a sense of connectedness
with family - with the parents who gave the name, and with others in a domestic
arrangement or a kin network with whom all or part of the name is shared."
\parencite[711]{finch08} We can find more subtle connecting functions, however.
Choosing a first name after an older ancestor connects you to a more specific
family relationship. And even the linguistic or religious connotations carried
within the first or last name can connect a person to or set them apart from the
dominant society in which they live.

\subsection{A Survey of the World's Names}

The form and contents of people's names vary immensely around the world.
Anglo-American practice entails two given names, the "first" and "middle" names,
being appended to a patrilineal surname. In many Hispanophone cultures, a child
receives both a patrilineal and a matrilineal surname, and the father's surname
taking precedence in terms of identification. An Icelandic surname consists of
the father's given name with the attached suffix -son or -dottir, depending on
gender. A similar practice occurs in Pakistan, but without any suffixation on
the father's given name. In South India many people have have three names, a
personal name, a family name, and a village name.

The ordering of a name's elements also varies. East Asian and Hungarian names
reverse the Western order, putting the given name after the family name.
Standard Chinese given names consist of two characters, whose meanings may or
may not be interconnected (Emma Woo). 
Even from a brief overview, it is clear that great diversity exists in names
worldwide; but we should not view each one as a static system. Instead, a naming
culture constitutes a space of social rules and expectations which allow for
cultural expression thru individual acts. On one end, a name can be used to
locate the individual within a subgroup of the society, due to the name's
linguistic, cultural, or religious connotations. For example, religion has been
an important influence on naming; bearing a Christian or Islamic name marks
someone as a likely member of that religious group. The linguistic origin of a
name may also convey information. People often ask the origin of my last name,
DeFreitas, which comes from Portuguese, a language I do not speak or have any
ancestral connection to. In other cases, a name's origin may alert you
successfully to the bearer's ancestry. In Kenya, we find the Giriama group,
whose clan name system identifies not only the bearer's clan, but also provides
information about the bearer' generation and birth order within the clan
\textcite{parkin89}.

Asking the cómo-se-llama in Western context presupposes a fixed answer. Imagine
the IRS' dismay if Walter demanded that he should addressed as Jean-Pierre for
the entirety of 2020. (It is, after all, a leap year.) The IRS would cease to
function if it had to honor such naming practices for 320 million citizens.
Administrative convenience, as \textcite{scott02} finds, has been the primary
factor promoting the spread of fixed legal names and hereditary surnames. For
example, in England, surnames were first adopted by the landed Norman élite. As
the bureaucracy strengthened and middling types aspired to emulate their status.
The practice spread down the social ladder and by the end of the 18th century
had reached all parts of Great Britain. An accelerated programme as such
occurred in Turkey starting in 1934 under the Westernising régime of Atatürk.
