\section{Functions of names}

Personal names have three main functions, which \textcite[392]{heymann11} calls
\textit{denotative}, \textit{connotative}, and \textit{associative}. The
canonical use of names is the denotative (or referential), as names serves as
linguistic token selecting an individual for discussion or address.
Connotative and associative information is equally present in names, as they
can convey cultural information about the named, such as nationality,
ethnicity, religion, etc, evoke a conception of the individual's personality,
or honour the legacy of a famous person or beloved ancestor. In some cases, the
two functions are opposed to one another, and a balance must be struck between
legibility and free expression.

\subsection{Name standardisation}

Name-giving is a cultural universal. Ethnographers have found no society past
or present that does not name individuals \parencite{alford87}. Naming is thus
deeply rooted as an aspect of human identity. Historians and ethnographers tell
us that preindustrial naming differs greatly from industrialised practices.
For example, Nuers in Sudan take two given names (maternal and paternal), a
ceremonial clan name, and an ox name of a favourite domestic animal
\parencite{wardhaugh92}. Likewise, nicknames are liberally bestowed for
personal traits and accomplishments and become widely used by community members
in groups like the Giriama of Kenya \parencite{parkin88}
\parencite{wardhaugh92}. The notion of names providing a fixed legal identity
across time and space is not a given, as names can and do vary greatly by
context.

With bureaucratisation and industrialisation comes a need for such fixing of
the name. A mobile population and impersonal state presence requires accurate
and speedy identification of individuals. Hence \textcite{scott02} observe that
patrilineal surnames tend to arise wherever centralisation is occurring: Qín
China, Norman England, Atatürkian Turkey. As they suggest, patrilineal surnames
are perfectly suited to state functions requiring clear identification of
individuals: taxation, conscription, and justice \parencite[18]{scott02}. In
fact, the ideal name from an administrative point-of-view is not a name at all,
but a unique serial number assigned to every person.\footnote{In the 1930s,
officials in the Canadian North distributed serialised dog tags to Inuits, to
alleviate identification difficulties due to their non-European names and
nomadic lifestyles.\parencite{scott02}}

The point of standardising names is to produce a particular style of knowledge
about the populace, which is legible, objective, and public. So the name can
become a tracker to follow the person's movements, marriages, kinship, and
careers, with the emphasis being placed on the individual's relation to society
rather than on the individual themselves. This is not to deny the real
importance of administrative objectives in legitimately defining the modern
role of names. Since names are the primary way of referring to other
individuals, they inherently take on a public function and as such must be
legible and usable for most people. But we should remain mindful of the
individual's traditional autonomy in naming and ensure that any abridgements
thereof by modern bureaucracies are limited and truly necessary for the public
interest.

\subsection{Symbolic functions}

\begin{aquote}{Janice "Lokelani" Keihanaikukauakahihuliheekahaunaele
	\parencite{lee-valley}}
	You see, to some people in the world, your name is everything. If I say my
	name to an elder Hawaiian (kupuna), they know everything about my husband's
	family going back many generations…just from the name.
\end{aquote}

The symbolic function of a name is the complex of information it conveys, be it
genealogical, cultural, linguistic, religious, etc. The name is a foundational
aspect of personal identity and an important choice for parents to make in
child-raising. Parents' motives in choosing a name vary widely, but in general,
baby names serve either to express some special aspirations that parents have
for the child or to honour a particular family relationship (or celebrity),
like a grandfather or aunt \parencite{finch08}. These two expressive aspects of
naming must themselves be balanced between what \textcite{finch08} calls
individualizing functions and connecting functions.

The individualizing aspects of naming are those aspects of a name which make a
statement on the individual themselves, including the associative and
connotative. The clearest manifestation of this is the choice of a child's
forename, although this choice clearly makes more of a statement on the parents
than the child itself: "In selecting a name (especially for a first-born child)
parents are not only determining the personhood of their child but are also
taking a key step in defining their own new identity as parents."
\parencite[718]{finch08} (cf. \parencite[399]{heymann11}) Hence a parent can
name their child something "beautiful" like "Isabella" or something "strong"
like "Samson", imparting the connotations of these names without any knowledge
of the newborn's future personality.

The related function of association entails a name that is the same as or
similar to that of a person whose memory would be honoured or evoked by use of
the name. In the case above, "Samson" associates the name's bearer with the
biblical hero. A name like "Adolf Hitler Campbell" associates one to an
infamous historical figure. 

Connecting functions are facets that locate the individual within larger
milieu.  This takes the form of surnames, which in Anglophone societies
identify the paternal family unit to which the individual belongs. "The
construction of a name, and its uses through a lifetime, also can embody a
sense of connectedness with family - with the parents who gave the name, and
with others in a domestic arrangement or a kin network with whom all or part of
the name is shared." \parencite[711]{finch08} Other connecting functions
include giving a first name after an older ancestor connects you to a more
specific family relationship or even the linguistic or religious connotations
carried within the first or last name, which can connect a person to or set
them apart from the dominant society in which they live.

\subsection{Legal perspectives}

The common law tradition is relatively lax on the form of names given at birth
or adopted by adults, as it recognises the right of individuals to adopt any
name they choose, unless chosen for a fraudulent purpose
\parencite[403]{heymann11}. Moreover, a person's name choice of name need not
be approved by officials, although individuals may seek court approval for its
legitimizing value. Rather a name becomes such through its regular use, since
common law considers personal names primarily for their denotative function,
and courts will look for examples of the name used in context to establish that
it indeed carries out the denotative function. For example, the New York
Superior Court rejected a name change from "Anatoly Eyzenberg" to "Tony
Eisenberg" for a City Council candidate who hoped to appear more American and
less Russian. The Court's decision rested on the fact that the petitioner had
not actually used the "new name" on any of his personal documents.
\parencite{eisenberg03} \parencite[404]{heymann11}.

Courts have a public interest in maintaining the usefulness and legibility of
names, although note that a total lack of ambiguity is neither possible nor
desirable, considering extremely common names, like "John Smith".  Despite
common law's hands-off approach toward names, American courts tend to intervene
when the denotative aspect of a name is inherently threatened by its content.
For example, in the case of a person named "Adolf Hitler Campbell", while the
name indeed functions as a reference, but "the connotations associated with
such names are so strong that they frustrate the ability of the names to server
their denotative function" \parencite[417]{heymann11}.  Moreover, courts are
reluctant to apply state "imprimatur" on the name change
\parencite[413]{heymann11} if the name is overly bizarre or offensive. For
example, an Ohio court denied an application for name change to "Santa Robert
Claus" because of the people's "proprietary right in the identity of Santa
Claus" \parencite[419]{heymann11} \parencite{handley00}, while a Pennsylvania
court denied "World Savior", noting that "no one has ever asked us to invoke
our legal authority to certify them as a savior of man"
\parencite[419]{heymann11} \parencite{bethea91}.
