\section{Names}

\begin{aquote}{Janice "Lokelani" Keihanaikukauakahihuliheekahaunaele}
You see, to some people in the world, your name is everything. If I say my name
to an elder Hawaiian (kupuna), they know everything about my husband's family
going back many generations...just from the name.
\end{aquote}

Nearly all human beings have one or more names. Article Seven of the Convention on
the Rights of the Child enshrines a person's fundamental right to a name: "The
child shall be registered immediately after birth and shall have the right from
birth to a name." \parencite{crc}

\subsection{Functions of a Name}

Why do people have names? We can first discern a top-level division in the
functions of names between \textit{reference} and \textit{symbolic} functions,
which categories we will further subdivide below.

\subsubsection{Reference Functions}

Consider a table that contains one row for every person with whom you are
acquainted. We store everything we know about a person in their database entry:
nationality, appearance, favorite foods, pet peeves, etc. But faced with this
vast dataset, we need each row (person) to have one relatively unique identifier
with which to access their entry and retrieve the other fields. (Using hair
color, for example, would be a poor choice. The datum "brown hair" does not
narrow our search to a single individual.) A personal name is this token by
which we single out a specific individual, a concise token which makes reference
to the larger concept of a given human being.

This reference function of a name operates independent of the name's content;
any reasonably unique token will allow us to navigate the index. There is no
reason that "John" is a more effective identifier than "12345678".

\subsubsection{Symbolic Functions}

The symbolic functions of a name are carried within the content of the name
itself. What interests us more here is the role played by the content of the
name and the complex of information it carries, be it genealogical, cultural,
linguistic, religious, etc. We can further subdivide a name's symoblic
functions; they sit on a spectrum between what \textcite{finch08} calls
individualizing functions and connecting functions.

Individualizing functions are those aspects of a name which make a statement on
the individual themselves. The clearest manifestation of this is the choice of a
child's forename; indeed this makes more of a statment on the parent than the
child itself. "In selecting a name (especially for a first-born child) parents
are not only determining the personhood of their child but are also taking a key
step in defining their own new identity as parents." \parencite[718]{finch08}
Hence a parent can name their child something "beautiful" like "Isabella" or
something "strong" like "Samson".

Connecting functions are facets that locate the individual within a larger
milieu. Most basically this takes the form of surnames, which in Anglophone
societies identify the paternal family unit to which the individual belongs.
"The construction of a name, and its uses through a lifetime, also can embody a
sense of connectedness with family - with the parents who gave the name, and
with others in a domestic arrangement or a kin network with whom all or part of
the name is shared." \parencite[711]{finch08} We can find more subtle connecting
functions, however. Choosing a first name after an older ancestor connects you
to a more specific family relationship. And even the linguistic or religious
connotations carried within the first or last name can connect a person to or
set them apart from the dominant society in which they live.

\subsection{Digital Names}

This paper explores a modern tension between the name's reference and symbolic
function: computers. As the world has become computerized and the world's
information stored in databases, At the same time, a name cannot remain the sole
province of its owner or family; it must facilitate interaction with the wider
world as a means of address and identification. If a name affirms your status as
an individual, it no less affirms your status as a citizen of your country,
resident of your city, customer of your electric service provider, holder of
your credit card, employee of your company, and recipient of your parking
ticket. A name is worth nothing if others people in the environs cannot
pronounce it, write it, or remember it.

\subsection{Computers}

In the "digital age" has changed writing from a free and individualized practice
to a one based in a discrete and logical structure consisting of discrete
glyphs. Since most early development of computers took place in the United
States, English gained a natural ascendancy over other languages in the field of
digital communication. English, perhaps as a coincidence, is also one of the
easiest languages to represent in code, requiring at the bare minimum just the
26 non-accented characters of the English alphabet, perhaps with some
punctuation and numbers. The ASCII standard encoding, with 127 available code
points, is more than enough to represent the English language in digital form.
Thus, organizations which deal only infrequently with non-English text have been
slow to update their databases to Unicode standards.
