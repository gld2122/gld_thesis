\section{Janice "Lokelani" Keihanaikukauakahihuliheekahaunaele}

In Hawaii, the only state in the United States having two official languages
(English and Hawaiian), a woman named Janice Keihanaikukauakahihuliheekahaunaele
has faced some administrative hassle due to her extensive surname. She was in
the news around 2013 for her battle against Hawaii's Bureau of Motor Vehicles
for an ID card which would include her full name. A recent renewal card she had
received included only her surname, with the final 'e' chopped off, which had
caused her an awkward situaiton at a traffic stop where the policeman questioned
her lack of a given name. After some complaining to the authorities, the state
was able to extend the the character limit from 35 to 40 characters and issue
Ms. Keihanaikukauakahihuliheekahaunaele a revised driver's licence.

To the bureaucrats handling her case, she was presumably an annoyance to whom
their first suggestion would be to change the name, perhaps back to her maiden
surname. Unfortunately Ms. Keihanaikukauakahihuliheekahaunaele did not
appreciate this suggestion, whose surname serves as a link to her late husband,
whose full name was Keihanaikukauakahihuliheekahaunaele, a traditional Hawaiian
name carrying genealogical information to those few who still understand
traditional Hawaiian language (supposedly cherished by the state). Presumably
this official recognition of the Hawaiian language was partly responsible for
the state's quick response to her request, as one can imagine the issue quickly
escalating from a media curiosity into a genuine debate about Hawaii's
commitmment to language and racial justice. We should contrast the successful
outcome of Ms. Keihanaikukauakahihuliheekahaunaele case with the uncoöperative
bureaucracy in China regarding the names of Ma Cheng and Zhao C. 
