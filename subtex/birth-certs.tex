\section{Birth Certificates}

American law generally holds the naming of children to be the right and
responsibility of parents, without shutting the door on regulating edge cases,
like "Ghoul Nipple", "Legend Belch",
"Brfxxccxxmnpcccclllmmnprxvclmnckssqlbb11116", and "" {[}Larson11{]}. In many
American states, however, this right is abridged with reference to diacritical
marks above letters, hardly an edge case in many languages around the world.
{[}Larson11{]} investigates this in his study of American naming law, finding
states with such rules to include California, Massachusetts, New Hampshire, and
Kansas:

\subsection{California}

Guidelines provided by the California Office of Vital Records instruct
county clerks that baby names may contain only "the 26 alphabetical
characters of the English language with appropriate punctuation if
necessary" and that "no pictographs, ideograms, diacritical marks
(including "é," "ñ," and "ç") are allowed" {[}Larson11{]}.

The handbook cites Proposition 63, the 1986 ballot referendum in which
Californian voters declared English the state's official language, as
legal justification. Larson points out, however, that the names of two
California state parks, Año Nuevo State Park and Montaña de Oro State
Park, manage to contain such characters. Moreover, the City of San José,
California includes the acute accented é in its official name, and its
Style Guide even includes instructions on how to produce it
electronically: "To create an accented é, hold down the alt key and
type "0233"‚ on the numeric key pad." California's Department of
Public Health likely disobeys the city's guidelines in birth
certificates, though this needs to be verified.

\subsubsection{Proposition 63}

\subsubsection{2014-AB-2528}

A 2014 bill in the California State Assembly sponsored by AM Nancy Skinner
(AB-2528) sought to rectify the state's processing of birth certificates and
driver's licenses by allowing diacritical marks in names. The bill "required the
State Registrar to ensure that diacritical marks on English letters are properly
recorded on birth certificates, death certificates, certificates of fetal death,
and marriage licenses, including, but not limited to, accents, tildes, graves,
umlauts, and cedillas" {[}AB-2528{]}. This bill stalled in the Appropriations
Committee when state agencies predicted multi-million dollar price tags relating
to IT upgrades, noting that the DMV's software could not "even accept lower-case
letters". For this same reason the bill was opposed by the County Recorder's
Association of California.

\subsubsection{2017-AB-82}

In 2017, California AM Jose Medina revived the issue with AB-82, which
ultimately passed both houses of the legislature before being vetoed by
Governor Jerry Brown. Unlike the 2014 bill, this edition did not affect
the issuance of driver's licenses, only birth certificates. Passing
through many more stages of the legislative process, the committee
hearings gathered more detailed estimates for the cost of IT upgrades
than they had in 2014:

\begin{itemize}
\item
  \$230,000 for IT upgrades at Department of Public Health
\item
  \$2 million per year for Department of Public Health to correct
  existing records
\item
  Loss of revenue of \$450,000 per year to Department of Public Health
  since they would not be able to electronically transmit names to SSA
  (at \$3 per name) containing diacritics
\item
  Up to \$12 million for local governments to upgrade their systems
\item
  \$1--3 million in upgrades to Department of Health Care Services
\item
  Unknown administrative costs to Department of Social Services
\end{itemize}

The sticking point for Governor Brown was compatibility with federal databases,
which do not accept diacritics. In his veto message, he argued that the risks to
vital records outweighed the benefits of cultural openness:

"Mandating the use of diacritical marks on certain state and local vital records
without a corresponding requirement for all state and federal government records
is a difficult and expensive proposition.  This bill would create
inconsistencies in vital records and require significant state funds to replace
or modify existing registration systems."

The committee findings make it clear that the state would incur nontrivial costs
to update the name registration systems. But no mention is made of the
possibility of finding creative solutions to the problem of encoding diacritics.
Even assuming that government systems cannot be made to support hte full UTF-8
standard, there are ways of representing information using ASCII. For example,
the international specification for machine-readable passports has a variety of
control sequences for representing subtle distinctions in the Latin, Cyrillic,
and Arabic alphabets using only the 26 plain characters of the English alphabet.
The original reform can be recovered nearly losslessly using the transliteration
table.

\subsection{Massachusetts}

In Massachusetts, the "characters have to be on the standard american
{[}sic{]} keyboard. So dashes and apostrophes are fine, but not accent
marks and the such". {[}Larson11{]}

\subsection{New Hampshire}

"All special characters other than an apostrophe or dash" are
prohibited. {[}Larson11{]}

\subsection{Kansas}

Restrictions are similar to those in Massachusetts. {[}Larson11{]}
