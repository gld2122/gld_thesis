\section{California}

California lies at the crossroads between American bilingualism and American
nativism. Due to its proximity to México and historical origin as Mexican
territory, the state has long been home to many Hispanophones.  During
California's 1848 annexation to the United States, a fraught issue was the
"Californios", Mexican residents of the territory, centred on the issue of
language. The Treaty of Guadalupe Hidalgo guaranteed linguistic rights for
Californian Hispanophones, reflected in the early legislature's diligent
publication of proceedings in English and Spanish. Legislative bilingualism
continued until (date), a turning point in California linguistic history, when
Spanish lost any claim to official status in the territory \parencite{baron02}.

Despite the Hispanic presence in the state, Californian voters approved several
"official English" ballot referendums in the twentieth century. A 1986 ballot
referendum, Proposiiton 63, declared English California's official language.
Proposition 63 created Article III, Section 6 of the California Constitution,
cementing the legal status of English and granting enforcement powers to the
state government: 

\begin{aquote}{California Constitution, Article III, Sec. 6(c)}
	The Legislature shall enforce this section by appropriate legislation. The
	Legislature and officials of the State of California shall take all steps
	necessary to insure that the role of English as the common language of the
	State of California is preserved and enhanced. The Legislature shall make no
	law which diminishes or ignores the role of English as the common language of
	the State of California.
\end{aquote}


\subsection{The Case of Spanish}

One modern battlefield for the tension between English and Spanish in California
is the birth certificate. Californian birth certificates allow only the 26
characters of English. American law consistently holds the naming of children to
be the right and responsibility of parents, disregardin edge cases, like "Ghoul
Nipple", "Legend Belch", "Brfxxccxxmnpcccclllmmnprxvclmnckssqlbb11116", and ""
\parencite{larson11}.  However, since About 29\% of California's population are
Spanish speakers, diacritical marks for Spanish names such as José are hardly an
edge case. \textcite[5]{larson11} investigates this in his study of American
naming law, finding California, Massachusetts, New Hampshire, and Kansas to be
among the states with such rules. We will focus here on California, because of
the sparsity of documentation in the other states and because California's large
Hispanophone population makes the situation there particularly glaring.

\parencite{acs-lang-states} Guidelines provided by the California Office of
Vital Records (OVR) inform county agents that baby names may contain only "the
26 alphabetical characters of the English language with appropriate punctuation
if necessary" and that "no pictographs, ideograms, diacritical marks (including
'é,' 'ñ,' and 'ç') are allowed" \parencite{larson11}.

The OVR's handbook cites Proposition 63, 
California's Department of Public Health interprets this language as mandating
the prohibition of "non-English" characters in Californian names; other
government agencies interpret the law differently. Two California state parks,
Año Nuevo State Park and Montaña de Oro State Park, manage to contain the
Spanish ñ in their official names, which is reflected on the parks' official
webpages. \parencite{año-nuevo} \parencite{montaña-de-oro} Likewise, the City of
San José, California includes the accented é in its official name, and its Style
Guide includes instructions on how to produce it digitally: "To create an
accented é, hold down the alt key and type '0233'‚ on the numeric key pad."
\textcite{san-josé} California's Department of Public Health likely disobeys the
city's guidelines in birth certificates, though this needs to be verified.

A 2014 bill in the California State Assembly sponsored by AM Nancy Skinner
(AB-2528) sought to rectify the state's processing of birth certificates and
driver's licenses by allowing diacritical marks in names. The bill "required the
State Registrar to ensure that diacritical marks on English letters are properly
recorded on birth certificates, death certificates, certificates of fetal death,
and marriage licenses, including, but not limited to, accents, tildes, graves,
umlauts, and cedillas". [ab-2528]

AB-2528 stalled in the Appropriations Committee once state agencies assigned
multi-million dollar price tags relating to IT upgrades, noting that the DMV's
software could not "even accept lower-case letters". For this same reason the
bill was opposed by the County Recorder's Association of California.

In 2017, California AM Jose Medina revived the issue with AB-82, which
ultimately passed both houses of the legislature before being vetoed by Governor
Jerry Brown. Unlike the 2014 bill, this edition did not affect the issuance of
driver's licenses, only birth certificates. Passing through many more stages of
the legislative process, the committee hearings gathered more detailed estimates
for the cost of IT upgrades than they had in 2014:

\begin{itemize}

\item \$230,000 for IT upgrades at Department of Public Health
\item \$2 million per year for Department of Public Health to correct existing
records
\item Loss of revenue of \$450,000 per year to Department of Public Health since
they would not be able to electronically transmit names to SSA (at \$3 per name)
containing diacritics
\item Up to \$12 million for local governments to upgrade their systems
\item \$1--3 million in upgrades to Department of Health Care Services
\item Unknown administrative costs to Department of Social Services

\end{itemize}

The sticking point for Governor Brown was compatibility with federal databases,
which do not accept diacritics. In his veto message, he argued that the risks to
vital records outweighed the benefits of cultural openness:

"Mandating the use of diacritical marks on certain state and local vital records
without a corresponding requirement for all state and federal government records
is a difficult and expensive proposition. This bill would create inconsistencies
in vital records and require significant state funds to replace or modify
existing registration systems."

The committee findings make clear that the state would incur nontrivial costs to
update the name registration systems. Little discussion is included of the
possible creative solutions to the problem. Even assuming that government
systems cannot be made to support the full UTF-8 standard, there are ways of
representing information using ASCII. For example, the international
specification for machine-readable passports has a variety of control sequences
for representing subtle distinctions in the Latin, Cyrillic, and Arabic
alphabets using only the 26 plain characters of the English alphabet.  The
original form can be recovered nearly losslessly using the transliteration
table.

In Massachusetts, the "characters have to be on the standard american keyboard.
So dashes and apostrophes are fine, but not accent marks and the such"
\parencite{larson11}.

"All special characters other than an apostrophe or dash" are prohibited
\parencite{larson11}. Technical limitations of the state's database systems
prevent the inclusion of any diacritical marks.

Restrictions are similar to those in Massachusetts \parencite{larson11}.
