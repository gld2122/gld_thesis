\section{California}

\subsection{Introduction}

California occupies a grey zone between American bilingualism and American nativism. In 2010, roughly 29\% of Californians spoke Spanish and 30\% of these speakers reported speaking English "not well" or "not at all" \parencite{acs-lang-states}. Meanwhile the historic significance of Spanish and Mexican culture to the state is inscribed in countless placenames like Los Angeles, San Francisco, and the Sierra Nevadas. Other than Spanish, California is home to speakers of East Asian languages like Tagalog, Chinese, Korean, and Vietnamese \parencite{acs-lang-states}. Yet none of these languages receive any recognition or protection at the state level, while English is enshrined in the state constitution as the official language of California \parencite{ca-const}.

The multilingual context of California makes its naming policies, as implemented by the Office of Vital Records, particularly unpalatable to many citizens. California registrars only accept the twenty-six letters of the English alphabet in names, meaning that no diacritic characters like "á", "é", or "ñ" are legally recognised in names. The ubiquity of Spanish names like José and María containing diacritic characters and large Hispanophone population makes California home to what is probably the United States' most extensive regulation of names. The diacritic ban is one in a long line of Official English policies in the state, a legacy which extends all the way back to its annexation into the United States.

\subsection{History of Language Attitudes in California}

During California's 1848 annexation from México into the United States the Treaty of Guadalupe Hidalgo promised equal rights to the approximately ten thousand Californios (the original Hispanophone community in the territory:

\begin{aquote}{Treaty of Guadalupe Hidalgo, Article IX}
	Mexicans who, in the territories aforesaid, shall not preserve the character
	of citizens of the Mexican republic, conformably with what is stipulated in
	the preceding article, shall be incorporated into the Union of the United
	States, and be admitted at the proper time (to be judged of by the Congress of
	the United States) to the enjoyment of all the rights of citizens of the
	United States, according to the principles of the constitution; and in the
	mean time shall be maintained and protected in the free enjoyment of their
	liberty and property, and secured in the free exercise of their religion
	without restriction. \parencite{guadalupe}
\end{aquote}

Enjoying "all the rights of citizens of the United States" did not turn out to include protection of Spanish. Although the original state constitution, promulgated with full Spanish translation, provided for publishing all legislation in both English and Spanish \parencite{baron92}. \textcite{lamar-prieto14} notes that even in the state's early days there were no Spanish courts, limiting legal representation for non-Anglophones. A contemporary observer wrote: "Si un Mexicano tiene por desgracia un pleito en las cortes de este Estado está seguro de perderlo" \parencite[28]{lamar-prieto14}. Legislative bilingualism continued only until the 1879 constitutional revision, when even the limited recognition of Spanish was discontinued, as supporters of the change reckoned that "California's Mexicans had had some thirty years to learn English" \parencite[]{baron92}. The 1879 constitution marked a turning point in California's linguistic history, as Spanish lost any claim to official status within California \parencite{baron92}.

\subsection{Proposition 63}

The early constitutional revision foreshadowed further "Official English" measures in the twentieth century. Official English refers to the a political movement beginning in 1980 promoting the imposition of English as official langauge at either the state or national level \parencite{liu14}. Supporters' motives range from the practical (economic necessity for immigrants to be conversant in English) to nationalistic (the spirit of American democracy ought be cherished in the original language of the Founders) to the racist \parencite[7]{baron92}.

In 1986 referendum, Proposition 63, voters declared English California's official language, creating Article III, Section 6 of the California Constitution and cementing the legal status of English and granting enforcement powers to the state government: 

\begin{aquote}{California Constitution, Article III, Sec. 6(c)}
	The Legislature shall enforce this section by appropriate legislation. The
	Legislature and officials of the State of California shall take all steps
	necessary to insure that the role of English as the common language of the
	State of California is preserved and enhanced. The Legislature shall make no
	law which diminishes or ignores the role of English as the common language of
	the State of California. \parencite{ca-const}
\end{aquote}

\subsection{Proposition 227}

In 1998, California voters approved Proposition 227, effectively ending bilingual education programs in the state. Classes taught in a bilingual setting would be replaced with nearly monolingual English classes designed for English learners.

\subsection{Birth Certificates}

A modern battlefield for official Spanish recognition in California is on the
birth certificate. Californian birth certificates allow only the 26 characters
of English. While American law (and common law tradition generally) holds the
naming of children to be the right and responsibility of parents, disregardin
edge cases, like "Ghoul Nipple", "Legend Belch",
"Brfxxccxxmnpcccclllmmnprxvclmnckssqlbb11116", and "" \parencite{larson11}.
However, diacritical marks for Spanish names like José are hardly an edge case.
\textcite[5]{larson11} investigates this in his study of American naming law,
finding California, Massachusetts, New Hampshire, and Kansas to be among the
states with such rules. We will focus here on California, because of the
sparsity of documentation in the other states and because California's large
Hispanophone population makes the situation there particularly glaring.

Guidelines from the California Office of Vital Records (OVR) instruct county
agents that names may contain only "the 26 alphabetical characters of the
English language with appropriate punctuation if necessary" and that "no
pictographs, ideograms, diacritical marks (including 'é,' 'ñ,' and 'ç') are
allowed" \parencite{larson11}.

The OVR handbook cites Proposition 63 as justification for banning diacritics.
California's Department of Public Health interprets the consitution's language
as prohibiting "non-English" characters in Californian names. Other government
agencies interpret the law differently. Two California state parks, Año Nuevo
State Park and Montaña de Oro State Park, contain the Spanish ñ in their
official names, which is reflected on the parks' official webpages
\parencite{año-nuevo} \parencite{montaña-de-oro}. Likewise, the City of San
José, California includes the accented é in its official name, and its Style
Guide includes instructions on how to produce it digitally: "To create an
accented é, hold down the alt key and type '0233'‚ on the numeric key pad."
\textcite{san-josé}

\subsection{Legislative Initiatives}

A 2014 bill in the California State Assembly sponsored by AM Nancy Skinner
(AB-2528) sought to rectify the state's processing of birth certificates and
driver's licenses by allowing diacritical marks in names. The bill "required
the State Registrar to ensure that diacritical marks on English letters are
properly recorded on birth certificates, death certificates, certificates of
fetal death, and marriage licenses, including, but not limited to, accents,
tildes, graves, umlauts, and cedillas". [ab-2528]

AB-2528 stalled in the Appropriations Committee once state agencies assigned
multi-million dollar price tags relating to IT upgrades, noting that the DMV's
software could not "even accept lower-case letters". For this same reason the
bill was opposed by the County Recorder's Association of California.

In 2017, California AM Jose Medina revived the issue with AB-82, which
ultimately passed both houses of the legislature before being vetoed by
Governor Jerry Brown. Unlike the 2014 bill, this edition did not affect the
issuance of driver's licenses, only birth certificates. Passing through many
more stages of the legislative process, the committee hearings gathered more
detailed estimates for the cost of IT upgrades than they had in 2014:

\begin{itemize}

\item \$230,000 for IT upgrades at Department of Public Health
\item \$2 million per year for Department of Public Health to correct existing
records
\item Loss of revenue of \$450,000 per year to Department of Public Health since
they would not be able to electronically transmit names to SSA (at \$3 per name)
containing diacritics
\item Up to \$12 million for local governments to upgrade their systems
\item \$1--3 million in upgrades to Department of Health Care Services
\item Unknown administrative costs to Department of Social Services

\end{itemize}

The sticking point for Governor Brown was compatibility with federal databases,
which do not accept diacritics. In his veto message, he argued that the risks
to vital records outweighed the benefits of cultural openness:

"Mandating the use of diacritical marks on certain state and local vital records
without a corresponding requirement for all state and federal government records
is a difficult and expensive proposition. This bill would create inconsistencies
in vital records and require significant state funds to replace or modify
existing registration systems."

The committee findings make clear that the state would incur nontrivial costs to
update the name registration systems.

\subsection{Analysis}

The use of the initiative process accords with the findings of \textcite{liu14}
that direct democracy increases the chance of official-English policies in
states with high immigrant populations.
