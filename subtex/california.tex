\section{California}

California occupies a crossroads between American bilingualism and American
nativism. As of 2010, roughly 29\% of California's population spoke Spanish ; of
these speakers, roughly 30\% report that they speak English "not well" or "not
at all" \parencite{acs-lang-states}. Meanwhile the Hispanic legacy is inscribed
in Spanish placenames across the state. Yet Spanish has no official recognition
in state law, while English is enshrined as official language in California's
constitution \parencite{ca-const}. With no diacritics accepted on California
birth certificates, the names of Californian citizens containing characters like
"á", "é", and "ñ" are misspelled by force of law. The ban on diacritics in names
makes California the site of America's most extensive name regulation.

\subsection{Annexation}

Due to the proximity to México and historical origin as Mexican territory, the
state has long housed many Hispanophones. At California's 1848 annexation to the
United States, one issue was how to guarantee the fair treatment of the (how many)
"Californios", Mexican residents of the territory. The Treaty of Guadalupe
Hidalgo promised equal rights for Hispanophones: 

\begin{aquote}{Treaty of Guadalupe Hidalgo, Article IX}
	Mexicans who, in the territories aforesaid, shall not preserve the character
	of citizens of the Mexican republic, conformably with what is stipulated in
	the preceding article, shall be incorporated into the Union of the United
	States, and be admitted at the proper time (to be judged of by the Congress of
	the United States) to the enjoyment of all the rights of citizens of the
	United States, according to the principles of the constitution; and in the
	mean time shall be maintained and protected in the free enjoyment of their
	liberty and property, and secured in the free exercise of their religion
	without restriction. \parencite{guadalupe}
\end{aquote}

Yet "enjoyment of all the rights of citizens of the United States" did not
entail protecting their native langauge. Although the original state
constitution provided for publishing legislation in both English and Spanish,
\textcite{lamar-prieto14} notes that the early state had no Spanish courts,
limiting legal access for non-Anglos. Legislative bilingualism continued only
until the 1879 constitutional revision, when bilingualism was removed, as Anglo
supporters of the change reckoned that "California's Mexicans had had some
thirty years to learn English" \parencite{baron92}. The new constitution
was a turning point in California linguistic history, when Spanish lost any
claim to official status in the territory \parencite{baron92}. 

\subsection{Proposition 63}

Despite a Hispanic legacy, Californian voters approved several "Official
English" referendums in the twentieth century. Official English refers to the
a political movement promoting the imposition of English as official langauge at
either the state (or preferably national) level. Supporters' motives range from the
practical (economic necessity for immigrants to be conversant in
English) to nationalistic (the spirit of American democracy ought be cherished
in the original language of the Founders) to the racist \parencite[7]{baron92}.

A 1986 ballot referendum, Proposition 63, declared English as California's
official language, creating Article III, Section 6 of the California
Constitution and cementing the legal status of English and granting enforcement
powers to the state government: 

\begin{aquote}{California Constitution, Article III, Sec. 6(c)}
	The Legislature shall enforce this section by appropriate legislation. The
	Legislature and officials of the State of California shall take all steps
	necessary to insure that the role of English as the common language of the
	State of California is preserved and enhanced. The Legislature shall make no
	law which diminishes or ignores the role of English as the common language of
	the State of California. \parencite{ca-const}
\end{aquote}

\subsection{Proposition 227}

In 1998, California voters approved Proposition 227, effectively ending
bilingual education programs in the state. Classes taught in a bilingual setting
would be replaced with nearly monolingual English classes designed for English
learners.

\subsection{Birth Certificates}

A modern battlefield for official Spanish recognition in California is on the
birth certificate. Californian birth certificates allow only the 26 characters
of English. While American law (and common law tradition generally) holds the
naming of children to be the right and responsibility of parents, disregardin
edge cases, like "Ghoul Nipple", "Legend Belch",
"Brfxxccxxmnpcccclllmmnprxvclmnckssqlbb11116", and "" \parencite{larson11}.
However, diacritical marks for Spanish names like José are hardly an edge case.
\textcite[5]{larson11} investigates this in his study of American naming law,
finding California, Massachusetts, New Hampshire, and Kansas to be among the
states with such rules. We will focus here on California, because of the
sparsity of documentation in the other states and because California's large
Hispanophone population makes the situation there particularly glaring.

Guidelines from the California Office of Vital Records (OVR) instruct county
agents that names may contain only "the 26 alphabetical characters of the
English language with appropriate punctuation if necessary" and that "no
pictographs, ideograms, diacritical marks (including 'é,' 'ñ,' and 'ç') are
allowed" \parencite{larson11}.

The OVR handbook cites Proposition 63 as justification for banning diacritics.
California's Department of Public Health interprets the consitution's language
as prohibiting "non-English" characters in Californian names. Other government
agencies interpret the law differently. Two California state parks, Año Nuevo
State Park and Montaña de Oro State Park, contain the Spanish ñ in their
official names, which is reflected on the parks' official webpages.
\parencite{año-nuevo} \parencite{montaña-de-oro} Likewise, the City of San José,
California includes the accented é in its official name, and its Style Guide
includes instructions on how to produce it digitally: "To create an accented é,
hold down the alt key and type '0233'‚ on the numeric key pad."
\textcite{san-josé}

A 2014 bill in the California State Assembly sponsored by AM Nancy Skinner
(AB-2528) sought to rectify the state's processing of birth certificates and
driver's licenses by allowing diacritical marks in names. The bill "required the
State Registrar to ensure that diacritical marks on English letters are properly
recorded on birth certificates, death certificates, certificates of fetal death,
and marriage licenses, including, but not limited to, accents, tildes, graves,
umlauts, and cedillas". [ab-2528]

AB-2528 stalled in the Appropriations Committee once state agencies assigned
multi-million dollar price tags relating to IT upgrades, noting that the DMV's
software could not "even accept lower-case letters". For this same reason the
bill was opposed by the County Recorder's Association of California.

In 2017, California AM Jose Medina revived the issue with AB-82, which
ultimately passed both houses of the legislature before being vetoed by Governor
Jerry Brown. Unlike the 2014 bill, this edition did not affect the issuance of
driver's licenses, only birth certificates. Passing through many more stages of
the legislative process, the committee hearings gathered more detailed estimates
for the cost of IT upgrades than they had in 2014:

\begin{itemize}

\item \$230,000 for IT upgrades at Department of Public Health
\item \$2 million per year for Department of Public Health to correct existing
records
\item Loss of revenue of \$450,000 per year to Department of Public Health since
they would not be able to electronically transmit names to SSA (at \$3 per name)
containing diacritics
\item Up to \$12 million for local governments to upgrade their systems
\item \$1--3 million in upgrades to Department of Health Care Services
\item Unknown administrative costs to Department of Social Services

\end{itemize}

The sticking point for Governor Brown was compatibility with federal databases,
which do not accept diacritics. In his veto message, he argued that the risks to
vital records outweighed the benefits of cultural openness:

"Mandating the use of diacritical marks on certain state and local vital records
without a corresponding requirement for all state and federal government records
is a difficult and expensive proposition. This bill would create inconsistencies
in vital records and require significant state funds to replace or modify
existing registration systems."

The committee findings make clear that the state would incur nontrivial costs to
update the name registration systems.
