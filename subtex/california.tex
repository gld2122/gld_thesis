\section{California}

\subsection{Introduction}

California occupies a grey zone between American bilingualism and American
nativism. In 2010, roughly 29\% of Californians spoke Spanish and 30\% of these
speakers reported speaking English "not well" or "not at all"
\parencite{acs-lang-states}. Meanwhile the historic significance of Spanish and
Mexican culture to the state is inscribed in countless placenames like Los
Angeles, San Francisco, and the Sierra Nevadas. Other than Spanish, California
is home to speakers of East Asian languages like Tagalog, Chinese, Korean, and
Vietnamese \parencite{acs-lang-states}. Yet none of these languages receive any
recognition or protection at the state level, while English is enshrined in the
state constitution as the official language of California \parencite{ca-const}.

The multilingual context of California makes its naming policies, as
implemented by the Office of Vital Records, particularly unpalatable to many
citizens. California registrars only accept the twenty-six letters of the
English alphabet in names, meaning that no diacritic characters like "á", "é",
or "ñ" are legally recognised in names. The ubiquity of Spanish names like José
and María containing diacritic characters and large Hispanophone population
makes California home to what is probably the United States' most extensive
regulation of names. The diacritic ban is one in a long line of Official
English policies in the state, a legacy which extends all the way back to its
annexation into the United States.

\subsection{History of Language Attitudes in California}

During California's 1848 annexation from México into the United States the
Treaty of Guadalupe Hidalgo promised equal rights to the approximately ten
thousand Californios (the original Hispanophone community in the territory:

\begin{aquote}{Treaty of Guadalupe Hidalgo, Article IX \parencite{guadalupe}}

	Mexicans who, in the territories aforesaid, shall not preserve the character
	of citizens of the Mexican republic, conformably with what is stipulated in
	the preceding article, shall be incorporated into the Union of the United
	States, and be admitted at the proper time (to be judged of by the Congress
	of the United States) to the enjoyment of all the rights of citizens of the
	United States, according to the principles of the constitution; and in the
	mean time shall be maintained and protected in the free enjoyment of their
	liberty and property, and secured in the free exercise of their religion
	without restriction.

\end{aquote}

Enjoying "all the rights of citizens of the United States" did not turn out to
include protection of Spanish. Although the original state constitution,
promulgated with full Spanish translation, provided for publishing all
legislation in both English and Spanish \parencite{baron92}.
\textcite{lamar-prieto14} notes that even in the state's early days there were
no Spanish courts, limiting legal representation for non-Anglophones. A
contemporary observer wrote: "Si un Mexicano tiene por desgracia un pleito en
las cortes de este Estado está seguro de perderlo"
\parencite[28]{lamar-prieto14}. Legislative bilingualism continued only until
the 1879 constitutional revision, when even the limited recognition of Spanish
was discontinued, as supporters of the change reckoned that "California's
Mexicans had had some thirty years to learn English" \parencite[]{baron92}. The
1879 constitution marked a turning point in California's linguistic history, as
Spanish lost any claim to official status within California
\parencite{baron92}.

\subsection{Proposition 63}

The early constitutional revision foreshadowed further "Official English"
measures in the twentieth century. Official English refers to a political
movement beginning in 1980 promoting the imposition of English as official
language at either the state or national level \parencite{liu14}. Supporters'
motives range from the practical (economic necessity for immigrants to be
conversant in English) to nationalistic (the spirit of American democracy ought
be cherished in the original language of the Founders) to the racist
\parencite[7]{baron92}.

In 1986 referendum, Proposition 63, voters declared English California's
official language, creating Article III, Section 6 of the California
Constitution and cementing the legal status of English and granting enforcement
powers to the state government: 

\begin{aquote}{California Constitution, Article III, Sec. 6(c)
	\parencite{ca-const}}

	The Legislature shall enforce this section by appropriate legislation. The
	Legislature and officials of the State of California shall take all steps
	necessary to insure that the role of English as the common language of the
	State of California is preserved and enhanced. The Legislature shall make no
	law which diminishes or ignores the role of English as the common language of
	the State of California.

\end{aquote}

\subsection{Birth Certificates}

With this historical context in mind, we come to our main focus regarding
California language policy, the longstanding restriction on any diacritic
characters from appearing in names on official documents. Guidelines from the
California Office of Vital Records (OVR) instruct local agents that names may
contain only the "26 alphabetical characters of the English language with
appropriate punctuation if necessary" and that "no pictograms, ideograms,
diacritical marks (including "é", "ñ", and "ç") are allowed"
\parencite{larson11}. The OVR handbook actually cites Proposition 63's'
official English provision as its justification for banning diacritics. Other
government agencies interpret the law differently. Two California state parks,
Año Nuevo State Park and Montaña de Oro State Park, contain the Spanish ñ in
their official names, which is reflected on the parks' official webpages
\parencite{año-nuevo} \parencite{montaña-de-oro} \textcite{larson11}.
Likewise, the City of San José, California includes the accented é in its
official name, and its Style Guide includes instructions on how to produce it
digitally: "To create an accented é, hold down the alt key and type '0233'‚ on
the numeric key pad." \textcite{san-josé}

\subsection{Legislative Initiatives}

Two recent bills in the California Assembly attempted to address the issue.
AB-2528 sponsored by AM Nancy Skinner in 2014 "required the State Registrar to
ensure that diacritical marks on English letters are properly recorded on birth
certificates, certificates of fetal death, and marriage licenses, including,
but not limited to, accents, tildes, graves, umlauts, and cedillas"
\parencite{ab2528}. The bill stalled in the Appropriations Committee due to the
high cost noted for IT upgrades. For example, the Department of Motor Vehicles'
software reportedly could "not even accept lower-case letters"
\parencite{ab2528}. The County Recorder's Association of California opposed the
bill for the same reason.

In 2017, AM Jose Medina (his official Assembly website does not include an é in
his first name \parencite{medina}) revived the issue with AB-82, which
ultimately passed both houses of the legislature before being vetoed by
Governor Jerry Brown, who again cited the high costs of upgrades. A break down
of those costs was presented in one committee hearing as follows
\parencite{veto}:

\begin{itemize}

\item \$230,000 for IT upgrades at Department of Public Health

\item \$2 million per year for Department of Public Health to correct existing
records

\item Loss of revenue of \$450,000 per year to Department of Public Health
since they would not be able to electronically transmit names to SSA (at \$3
per name) containing diacritics

\item Up to \$12 million for local governments to upgrade their systems

\item \$1--3 million in upgrades to Department of Health Care Services

\item Unknown administrative costs to Department of Social Services

\end{itemize}

The cost and compatibility issues with federal systems was the sticking point
for Governor Brown. His veto message argues that the costs and risks are too
great to justify the bill:

\begin{aquote}{\parencite{veto}}
	
	"Mandating the use of diacritical marks on certain state and local vital
	records without a corresponding requirement for all state and federal
	government records is a difficult and expensive proposition. This bill would
	create inconsistencies in vital records and require significant state funds
	to replace or modify existing registration systems."

\end{aquote}

\subsection{Analysis}

The common-law foundation of American naming law is the near-total right of
parents to name their children \parencite{heymann11}. The type of statutory
naming codes we will see later in Iceland and Latvia are unlikely o occur
outside of a civil law context. Instead any name which is not used for
fraudulent or malicious purposes should be accepted. \parencite{ferner96}
\parencite{finch08} \parencite{heymann11}. California's diacritic ban, however,
should be viewed as naming law in the traditional common-law conception. It is
instead an act of linguistic policy pursuant to Proposition 63. That is, no one
presumably considers the name María to be inherently incompatible with the
denotative function required in law for names. It is instead contrary to the
government's supposed interest in protecting the use of English in the United
States.

Both \textcite{larson11} and \textcite[598]{foggan83} find that child-naming must
be considered a "constitutionally protected child-rearing decision". This
places it in the purview of the Fourteenth Amendment's Due Process clause,
which has been interpreted to include a "right to privacy" in parental
decisions\footnote{A right to privacy is articulated explicitly in more modern
human rights documents, such as Article 17 of the International Covenant on
Civil and Political Rights \parencite{iccpr}}. \textcite{larson11} notes that
it is unclear what standard courts should use in balancing individual rights
against legitimate state interests (eg. rational basis vs. strict scrutiny).

Considering the diacritic ban in California under Due Process would entail
recognising the inherent capacity of parents to make choices for their
children's names and cultural upbringings, keeping in mind the common-law
tradition of onomastic freedom, as well. No legitimate state interest could be
raised based on the child's best interest in prohibiting the use of names like
José. If anything, it is in the child's best interest to keep his or her
cultural identity intact. Moreover, even the argument of "legibility", or the
interest of courts in maintaining the public usability of names, is unsound, as
such names, being written in Roman letters, are readily comprehensible to
Californian Anglophones, and there is no legal requirement that
\textit{everyone} write the diacritic \textit{every time}. As
\parencite{larson11} notes, "Costco...annoy, etc".

Even assuming that the stated goals of Proposition 63 are legitimate in
supporting English as the official language, the measure may not be upheld, as
it appears more designed to disenfranchise Spanish than to bolster English: "it
is hard to see why permitting the name `Changsurirothenothenom' while
prohibiting `Lucía' serves any state interest in promoting the English
language" \parencite[189]{larson11}. Besides, the ability of states to
interfere in parent's private language decisions is thrown into doubt by the
landmark 1923 Supreme Court case \textit{Meyer v. Nebraska}, which recognised
the child's right to instruction in a foreign language \parencite{larson11}
\parencite{baron92}. The diacritic ban raises similar problems, as giving
children a properly spelled surname in the heritage language is an aspect of
passing on cultural heritage.

The First Amendment offers another potential vehicle for legal argumentation,
as namegiving is of speech. \textcite{larson11} notes that this line of thought
offers protection to a wider spectrum of names (ie. pets' names)
\parencite{larson11}. Note, however, \textit{re}, in which the court found that
``Petitioner has a right under the common law to assume any name that he wants
so long as no fraud or misrepresentation is involved \dots once Petitioner
files an application for a name change \dots and seeks the approval of the
courts for a name, it becomes the responsibility of the courts to ensure that
there are no lawful objections to the name change'' \parencite{variable08}
\parencite[413]{heymann11}. Thus this New Mexico court adopted the "imprimatur"
argument noted by \textcite{heymann11}.

The remaining argument available to the state is that the computer systems
currently in use cannot handle the addition of diacritics and that upgrades
would be prohibitively expensive. Although the argument has some force, in that
governments may legitimately ensure the ability of names to function properly
in the public realm, diacritic marks do not inhibit comprehension of the name
and can always be omitted if necessary for a particular application. For
important and canonical records like a birth certificate, however, it seems
counterintuitive that the technological advancement in computing technology
should turn out to limit citizens' individual rights in comparison to a
handwritten and typewriter age \parencite[191]{larson11}. Per the decision in
\parencite{ferner96}, which upheld the legality of a woman's name change to the
single word ``Koriander'', "computers and record keepers need not control
individual liberties".

California is not alone among US states in banning diacritical marks, as at
least three others, Kansas, Massachussetts, and New Hampshire have the same
policy. Doubtless, additional states omit diacritics as an unofficial
administrative practice. California's situation is notable for the huge number
of people it affects in the name-giving process, given the state's prominent
Spanish-speaking population and Mexican legacy. Even if California's Official
English provision (Proposition 63) is constitutional, the administrative
interference in naming contravenes private naming rights under Due Process. For
an international comparison, see below on Latvia, which considers a similar
question of private naming rights from the perspective of international human
rights.
