\section{Iceland}

Iceland's naming system is distinctive for retaining its patronymic (non-fixed)
surnames and the government's maintenance of a Personal Names Register for
child-naming. Both policies reflect Iceland's insular nationalism and pride for
its conservative North Germanic language. Like Latvia, Iceland's active state
promotion of language preservation occurs in a postcolonial context, as it
asserted its unique heritage and emerged from the linguistic shadow of Danish
rule. Controlling the spread of modern-type fixed surnames, which threatened to
displace the indigenous patronymic system, was seen as a way of maintaining
Icelandic identity \parencite{willson02}. The institution of the Personal Names
Register also follows this logic and represents an effort to maintain the
cultural and grammatical structure of Icelandic. However, the appropriateness
of Icelandic regulations should be considered from the human rights perspective
as well, especially in the wake of the UN decision in Latvia discussed above.

\subsection{Icelandic Nationalism and Purism}

Nationalist and Romantic tendencies gained strength in Iceland in the second
half of the nineteenth century in the context of Iceland's struggle for
independence from Denmark. Icelanders claimed their unique linguistic heritage
as one justification for political independence. The ideology of linguistic
purism went virtually unchallenged within the general populace.
\textcite{kristinsson12} notes that there was harmony between "practices,
beliefs, and management decisions" in language policy during Iceland's period
of peak nationalism between 1860 and 1960.

A shift in focus occurred in Icelandic linguistic policy in the 1960s, when the
state created the Icelandic Language Council. The postwar presence of US forces
at a permanent military installation marked the end of Danish supremacy and the
beginning of Anglophone cultural imperialism. \textcite{kristinsson12}
interprets the creation of the Council as a reaction to the more fragile
position of Icelandic, as linguistic policymakers sought to stem the tide of
global cultural influence on Icelandic.

\subsection{Surname Debate}

Although Icelandic clearly faced an unprecedented global threat in the
aftermath of World War II, we should note that this phenomenon was foreshadowed
even around the turn of the century in the debate over whether to adopt modern
fixed surnames, and that even during this time, linguistic purism ideology was
not maintained entirely without legal intervention. One of Icelandic's salient
characteristics is the patronymic system of surnames. The majority of
Icelanders take as a surname their father's or mother's given name, in the
genitive case and with the appropriate "-son" or "-dóttir" suffix attached,
depending on gender.

Wilson notes that the preservation of this unique system was not inevitable,
and policymakers and linguists at the turn of the century debated whether
Iceland should adopt fixed surnames. Despite the linguistic purism of much of
Iceland's population, this debate was responding to a real surge of
Continental-style surnames in the nineteenth century, as persons from the
upwardly mobile class pursued studies in Denmark or elsewhere in Europe and
brought the practice home to Iceland. Ultimately, the spread of surnames was
only prevented via legal action.

The first attempt to limit surname adoption was an 1881 bill introduced in the
Alþingi instituting a tax on the adoption of such names. Thus the proposal
would have required royal permission to adopt a surname, and if approved, would
require the payment of five hundred; hereafter the nameholder would also be
subject to an annual title tax ("nafnbótarskattur") of ten crowns per syllable
in the name \parencite[137]{willson02}. This proposal was not approved,
however, and the debate raged on about what should be done

To briefly summarise the debate presented in-depth by \textcite{willson02}, we
can note that the two camps roughly represented the modernist and conservative
perspectives. Writer Guðmundur Kamban presents an internationalist defence of
surnames (he himself used a fixed surname), claiming that female Icelanders
abroad will inevitably be referred to by their husband's patronymic, creating a
gender mismatch, and moreover praises the extreme morphological simplification
of English as a sign of linguistic development. Countering Guðmundur's
forward-thinking approach, the conservative line was carried by linguist
Jóhannes Jóhannsson, who touted the linguistic heritage of Icelandic and
suggested that, if surnames should prove inevitable, they should be made more
harmonious with the language by incorporating more authentic Icelandic
elements.

A law on the topic finally passed the Alþingi in 1913, levying a moderate
charge for surname adoption without any annual tax. A government report in
1915, \textit{Íslenzk mannanöfn} (Icelandic Personal Names), carried on the
work of Jóhannes in trying to develop an Icelandic adaptation of modern
surnames, going to great lengths to justify their recommendations with sound
philological argumentation. Still, the report emphasised the need for
modernisation, as the writers chose not to include any surnames including the
letter Þ, due to its likelihood for corruption in foreign contexts.

Apparently this report stretched the linguistic purism of Iceland's population
to the breaking point. The report's suggestions were ridiculed by writers as an
indication that surnames were inherently incompatible with Icelandic. The
counterreaction was consummate in 1925, when the adoption of new surnames was
firmly forbidden by the Alþingi. Although ultimately the patronymic system
remained intact, the Icelandic surname debate casts doubt on the inevitability
of maintaining Iceland's purity. Legal intervention was necessary to control
the public's naming practices and prevent a structural change of the national
language.

\subsection{Personal Names Committee}

A better-known aspect of Iceland's naming policy is the Personal Names
Committee (Mannanafnanefnd). The committee of three people maintains a registry
of permissible forenames that are "capable of having Icelandic genitive
endings", "written in accordance with the ordinary rules of Icelandic
orthography", and gender-conforming: "girls shall be given women's names and
boys shall be given men's names" (Personal Names Act 1996). If a parent wishes
to use a name absent from the list, they must make a request to the Committee
for approval. Along with their ruling (which is explicity non-appealable under
the Personal Names Act), the Committee publishes an explanation of their
decision. As a typical example of their reasoning, "Huxland" was rejected in
2014 because there was no precedent for geographical suffixes in given names
(only in middle names!) and because the suffix "-land" implies a neuter noun
(Kyzer).

The most famous case from the Naming Committee is that of Blær Bjarkardóttir
Rúnarsdottir, whose first name was rejected by the Committee, since they
classified it as a male name. Blær's mother had been mistakenly informed by a
priest that the name was acceptable (AP). However, the family refused to back
down and choose a new name. Thus the state applied the generic forename Stúlka
("girl") to all of Blær's official documents, including her passport. Clearly,
Iceland takes enforcement of naming policies seriously. Only after arguing that
the name Blær was used for a female character in a popular Icelandic novel did
Reykjavík District Court allow the name to be used.

\subsection{Human Rights Implications}

Iceland's practice of labelling children as "Stúlka" (girl) or "Drengur" (boy)
if their parents do not comply with the Names Act raises concerns under human
rights, as this could be considered demeaning and disproportionate to the
support of an official language. A generic placeholder name is arguably no name
at all, and this would constitute a violation of Article Seven of the
Convention on the Rights of the Child, which states that all to children have
the right to a name and to which Iceland is a signatory. This being the most
bare-bones conception possible of naming rights, Iceland's violation should be
addressed by the international community.

In the light of the United Nations Human Rights Committee's decision in
\parencite{raihman10}, it is difficult to see how Iceland's policy, if
challenged in this forum, would not be struck down accordingly. Like the case
in Latvia, Iceland's Personal Names Act circumscribes individuals' choice of
names in order to support the official language. Recalling that Article 17 of
the International Covenant on Civil and Political Rights encompasses the "right to
choose and change one's own name", Iceland's policy contravenes the state's
obligations under this provision. Despite Iceland's legitimate interest in
protecting their national language, it is imperative that linguistic management
does not interfere with individual freedoms.
