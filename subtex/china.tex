\section{China}

\subsection{Introduction}

Given the People's Republic of China's (PRC) diverse population and linguistic
landscape, language has served as an important vehicle of control for the
Communist government. The PRC has distinguished mainland Chinese from the
language of Taiwan via the simplification of Chinese script in the 1950s,
reducing both the number and complexity of characters. Moreover, PRC Chinese is
distinctive by its standardised use of Hànyǔ Pīnyīn since 1958, vis-à-vis
Taiwan, where competing systems like Wade-Giles continue to to coëxist with
Pīnyīn. Government intervention favouring a nationwide Mandarin standard
(Pǔtōnghuà: "common language") puts Chinese provincial languages at a
disadvantage, including Yuè and Wú. Pǔtōnghuà is "enregistered" as the
accentless variety \parencite{dong10}, while regional dialects are endangered as
upwardly-mobile youth move to cities and adopt the standard variety. In 2003,
Xing found that there were 22 minority languages with fewer than 10,000
speakers.

\subsection{Second-Generation ID Cards}

In 2004, China's Public Security Bureau (PSB) instituted a computerised system for
issuing and tracking identification information. The new ID cards were intended
to improve the PSB's capacity to compile population statistics and offer more
robust anti-forgery protection. The digital management of the ID system,
however, caused administrative hassle for some Chinese citizens with uncommon
names. The reason is that government computers can only accommodate some 32,252
characters (art.), and if someone's name includes a character outside this set,
they are told to change it.

\subsection{Mǎ Chěng}

Mǎ Chěng ({\zafont 马}{\zbfont 𩧢}) was one such victim of the policy, as reported
in a \textit{New York Times} article by \textcite{lafraniere09}. Her grandfather
scoured a classical Chinese dictionary for a distinctive name. He settled on an
obscure character which complemented the family name mǎ ({\zafont 马}) "horse":
chěng ({\zbfont 𩧢}) "galloping steeds", or the character {\zafont 马马马} written
three times. Her previous PSB ID included a handwritten {\zbfont 𩧢}, but the new
computerised system would not allow this, and the Bureau recommended she change
her name to become compatible with the database.

The Chinese government's treatment toward Mǎ's name reveals its ambiguous
relationship with the Classical legacy. On one hand, their promotion of
Pǔtōnghuà is presented in terms of the Confucian ideal of hé ({\zafont 和}),
especially under former President Hú Jǐntāo's Harmonious Society programme
\parencite{wang16}. Yet the Communist Party's ongoing support for simplified
script and control of the official "character list" favours progress over the
veneration of history.

\subsection{Zhao C}

The same ID cards caused an administrative hassle for Zháo C ({\zafont 赵}C). The
PSB ruled that only Chinese characters may appear in names on the
second-generation cards. As Zháo's father is a lawyer, they appealed the
decision, arguing that 'C' appears in Pīnyīn romanisations, and in the
tradenames of large Chinese institutions such as CCTV, the Chinese national
broadcasting service. Apparently the judge was unconvinced, as Zháo volunatarily
agreed to change his name, asking the public for suggestions.

Unlike in Mǎ Chěng's situation, the letter 'C' is easily representable on
Chinese computers; instead of an administrative contingency, the restriction on
English letters intendes to protect the Chinese linguistic framework from
Western influence. In this way, China's government is taking an approach more
typical of small countries protecting their national language, as we see in the
case of Iceland and Lithuania. Indeed, as Zháo pointed out, infiltration of
English into Chinese society (CCTV, for instance) has occurred, despite the
simultaneous spread of Pǔtōnghuà as a hegemonic dialect in China. China's (and
Lithuania's) specific ban on foreign influence in names, contrasted with the
relative lenience on matters like signs and business names ("taxi", "TV")
suggests that names are considered very close to "the language itself". In other
words, speakers think of names as exemplifying the social content and meaning of
a language, and thus in need of protection from foreign influence. Yet in China,
perhaps uniquely, the name régime acts against threats from both the past (the
conservative veneration of Classical Chinese) and the future (the globalisation
of English), seeking a balance between the two approaches.


