\section{China}

\subsection{Introduction}

Given the People's Republic of China's (PRC) diverse population and linguistic
landscape, language is an important vehicle for promoting the government's
cultural policy. The PRC has distinguished mainland Chinese from the language of
Taiwan via the simplification of Chinese script in the 1950s, reducing both the
number and complexity of characters. Moreover, mainland Chinese is distinctive
for the standardised use of Pīnyīn romanisation since 1958, vis-à-vis Taiwan,
where competing systems like Wade-Giles continue to to coëxist with Pīnyīn.
Government intervention favouring a nationwide Mandarin standard (Pǔtōnghuà:
"common language") puts both provincial Chinese dialects (Cantonese,
Shanghainese) and non-Chinese languages (Tibetan, Uyghur) at risk. Pǔtōnghuà is
thus "enregistered" as an "accentless" dialect \parencite{dong10}, while
regional dialects are in danger as upwardly-mobile youth move to cities and
adopt the standard variety. In 2003, Xing found that there were 22 minority
languages with fewer than 10,000 speakers.

\subsection{Second-Generation ID Cards}

China's Public Security Bureau (PSB) instituted a digital system around 2004 to
issue and track personal identification cards. The new IDs have an embedded
microchip with personal and security information and were intended to improve
the Bureau's population statistics and anti-forgery protection
\parencite{ciicn04}.

Unfortunately, digital management of the ID cards caused administrative hassle
for many Chinese citizens with uncommon names. Government computers (as of 2009)
can only accommodate 32,252 Chinese characters, so if someone's name includes a
character outside the list, they are asked to change it
\parencite{lafraniere09}. Due to the open-ended nature of Chinese script, it is
possible to find legitimate characters that are barely comprehensible to the
average Chinese reader, due to their being uncommon or archaic. Most hànzì
consist of smaller radicals, which are themselves hànzì or derivatives thereof.
Thus the writing system is composite and open to user manipulation, unlike
alphabetic systems. There are two ways of seeing this feature of the script. As
Chinese has comparativley few surnames, some worry that name trends could take
parents into the Classical Chinese abyss and away from the realm of
comprehension: A linguistics professor interviewed by \textcite{lafraniere09}
noted: "The computer cannot even recognize them and people cannot read them.
This has become an obstacle in communication." On the other, Mǎ had lived
satisfactorily with the name for twenty-six years. Whenever necessary, she
explained her name to people. Agencies were often willing to accommodate the
rare character, by writing it by hand, writing it in pīnyīn, searching the
Unicode table, or even building the character themselves with software
\parencite{martinsen08}. When ID cards were handwritten, such a problem could
not arise.

\subsection{Mǎ Chěng}

Mǎ Chěng ({\zafont 马}{\zbfont 𩧢}) was one victim of this policy, as reported
by \textcite{lafraniere09} for \textit{The New York Times}. Mǎ's grandfather
scoured a classical Chinese dictionary for a distinctive name, settling on an
obscure character which complemented the family name mǎ ({\zafont 马}), meaning
"horse": chěng ({\zbfont 𩧢}) "gallop" (the traditional character
{\zafont 馬馬馬} written three times). Her previous ID card included a
handwritten {\zbfont 𩧢}, but the computerised system would not allow this,
and the Bureau recommended she change her name to become compatible with the
database.

The government's treatment of Ma's name suggests an ambiguous relationship with
the Classical Chinese legacy. On one hand, promotion of Pǔtōnghuà is presented
in terms of the Confucian ideal of hé ({\zafont 和}) "harmony", under former
President Hú Jǐntāo's Harmonious Society programme \parencite{wang16}. Yet the
CPC's ongoing efforts to simplify and control the Chinese script favours
progress over the veneration of history.

\subsection{Zhao C}

Zháo C ({\zafont 赵}C) faced a similar situation due to an English letter
appearing in his name. The Bureau ruled that only Chinese characters may appear
in names and refused to issue a second-generation card to Zháo. Since Zháo's
father is a lawyer, they appealed the decision, arguing that 'C' appears in
Pīnyīn romanisations and in tradenames of Chinese institutions like CCTV, the
Chinese national broadcasting service. Apparently the judge was unconvinced, as
Zháo volunatarily agreed to change his name, asking the public for suggestions.
\parencite{martinsen09}

Although the letter 'C' is easily represented on Chinese computers the
restriction on English letters intends to protect the Chinese linguistic
framework from Western encroachment. In this way, China's approach is more
typical of small countries protecting their national language, as we see in the
case of Iceland and Lithuania. As Zháo pointed out, infiltration of English into
Chinese society (CCTV, for instance) has occurred, despite the simultaneous
spread of Pǔtōnghuà as a hegemonic dialect in China. China's (and Lithuania's)
specific ban on foreign influence in names, contrasted with the relative
lenience on matters like signs and business names ("taxi", "TV") suggests that
names are considered very close to "the language itself".

\subsection{Analysis}

In both Lithuania and China, personal names are subject to more scrutiny than
the language at large. Borrowed words like "taxi" or "TV" are accepted when used
in a commercial context, but unacceptable when used in a name. Speakers think of
names as exemplifying the social content and meaning of a language variety, and
thus in need of protection against foreign influence. Yet in China, perhaps
uniquely, the name régime acts against threats from both the past (the
conservative veneration of Classical Chinese) and the future (the globalisation
of English), seeking a balance between the two approaches.
