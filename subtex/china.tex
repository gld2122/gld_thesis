\section{China}

\subsection{Introduction}

Given the People's Republic of China's diverse population and linguistic
landscape, language provides the state with an important vehicle for pursuing
cultural policies. The PRC has distinguished mainland Chinese from the language
of Taiwan via the simplification of Chinese script in the 1950s, which reduced
both the number and complexity of characters. Moreover, mainland Chinese is
distinctive for the standardised use of Pīnyīn romanisation since 1958,
vis-à-vis Taiwan, where competing systems like Wade-Giles continue to to
coëxist with Pīnyīn. 

The Communist government has sponsored a nationwide Mandarin standard in such
bills as \textit{Actively Promote the Beijing-based Putonghua as the Standard
Pronunciation} (1955) and \textit{Law of the National Common Language for the
People’s Republic of China} (2000) \parencite{dong10}. Linguistic intervention
in favour of a standard language (Pǔtōnghuà: "common language") puts both
provincial Chinese dialects (Cantonese, Shanghainese) and non-Chinese languages
(Tibetan, Uyghur) at risk. Pǔtōnghuà is thus "enregistered" as an "accentless"
dialect \parencite{dong10}, while regional dialects are in danger as
upwardly-mobile youth move to cities and adopt the standard variety.

Here we look at two recent naming cases in China, which stem from the
digitisation of the central identification system in the early 2000s and
evaluate how the policy aligns with the country's language policy in general.
This relates to the limitations of representing names in a digital format but
also reflects the homogenising effect of state infrastructure. Here we see both
the desire to propagate a standardised and progressive Mandarin and to avoid
the intrusion of global English.

\subsection{Second-Generation ID Cards}

China's Public Security Bureau instituted a digital system around 2004 to issue
and track personal identification cards. The new IDs have an embedded microchip
with personal and security information and were intended to improve the
Bureau's population statistics and anti-forgery protection \parencite{ciicn04}.

The change did not occur without drawbracks. The digital system caused
administrative hassle for many Chinese citizens with uncommon names. Government
computers (as of 2009) can only accommodate 32,252 Chinese characters. If
someone's name includes a character outside the list, they are asked to change
it \parencite{lafraniere09}. A woman named Mǎ Chěng ({\zafont 马}{\zbfont 𩧢})
was one victim of this policy, as reported by \textcite{lafraniere09} for
\textit{The New York Times}. Mǎ's grandfather scoured a classical Chinese
dictionary for a distinctive name and chose an obscure character, which
complemented the family name, mǎ ({\zafont 马}) "horse". The given name he
chose was chěng ({\zbfont 𩧢}) "gallop" (the traditional form of {\zafont 马}
({\zafont 馬}) written three times). Mǎ's previous ID card included a
handwritten {\zbfont 𩧢}, but the digital system would not allow this. Thus the
Bureau recommended she change her name for compatibility purposes.

The open-ended nature of Chinese script makes it possible to find legitimate
characters that are barely comprehensible to the average reader, due to their
being uncommon or archaic. Most characters consist of smaller radicals, which
are themselves characters or derivatives thereof. Thus the writing system is
composite and open to internal manipulation, unlike alphabetic systems. This
fact can be viewed positively or negatively.

As Chinese uses comparatively few surnames, some worry that name trends will
carry parents down into the Classical Chinese abyss, away from the realm of
mutual comprehension. For example, linguistics professor interviewed for the
\textcite{lafraniere09} article notes: "The computer cannot even recognize
[some names] and people cannot read them. This has become an obstacle in
communication." On the other hand, Mǎ lived satisfactorily with the name for
twenty-six years. Whenever necessary, she explained her name to people.
Agencies were often willing to accommodate the rare character by handwriting
it, writing in pīnyīn, searching the Unicode table, or even building the
character themselves with software \parencite{martinsen08}. In the process,
those whom Mǎ encounters learn more about their own writing system and its
history. This give-and-take (in any naming society) is lost with the onset of
rigid administrative rules.

A similar situation occurred in the case of Zháo C ({\zafont 赵}C) due to the
English letter that appears in his name. The Bureau ruled that only Chinese
characters may appear in names and refused to issue him a second-generation
card. Zháo and his father (a lawyer) appealed the decision, arguing that `C'
appears in Pīnyīn romanisations and in the tradenames of Chinese institutions
like CCTV, the Chinese national broadcasting service. Apparently the judge was
unconvinced, as Zháo volunatarily agreed to change his name, asking the public
for their suggestions \parencite{martinsen09}.

\subsection{Remarks}

The government's treatment of Ma's name suggests an ambiguous relationship with
the Classical Chinese legacy. Promoting a nationwide Pǔtōnghuà is presented in
terms of the Confucian ideal of hé ({\zafont 和}) "harmony", under former
President Hú Jǐntāo's Harmonious Society programme \parencite{wang16}. Yet the
CPC's ongoing efforts to simplify and control the Chinese script favours
progress over stability and apparently shuns the unifying force of the
Classical past.

Although the letter `C' is easily represented on Chinese computers, the
restriction on English letters in names intends to protect the Chinese
linguistic framework from Western encroachment. As such, China's approach is
actually typical of small countries protecting their national language, as we
see in the case of Iceland and Latvia. As Zháo pointed out, infiltration of
English into Chinese society (CCTV, for instance) has occurred, despite the
simultaneous spread of Pǔtōnghuà as a hegemonic dialect in China. Of course,
one might debate the relative likelihood of Mandarin Chinese's obolescence at
the hands of English. Moreover, we note that China's specific ban on foreign
influence in names, contrasted with the relative lenience on matters like signs
and business names ("CCTV") suggests that names are considered very close to
"the language itself".

Thus China's naming policies appear in some senses contradictory, attempting to
counteract the looming threat of global English, but simultaneously, as the
world's largest country and language, attempting to produce its own hegemonic
environment as a unifying language for China's many ethnic identities. Chinese
must fight a two-front war against both the past (conservative veneration of
Classical Chinese) and the future (global English), seeking a balance between
the two approaches.
