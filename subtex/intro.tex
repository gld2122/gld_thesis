\section{Introduction}

Since the 1960s, computers have assumed a vital role in ensuring the smooth
functioning of modern society, through their ability to store and process data
with astonishing efficiency. The impact is felt in all types of endeavours and
in all regions of the world, developed and developing alike. As such, computing
sits at the crossroads of many pressing social discourses, such as centralisation,
globalisation, alienation, etc. Clearly it would be simplistic to attribute
these broad processes solely to a change in technology; the centralisation of
state power, for instance, has been ongoing since at least the High Middle Ages
in the case of Western Europe. Nonetheless, the computer acts in the name of
ruthless efficiency, accelerating the spread of such phenomena.

Still, the victims of centralisation and globalisation find ways to harness
technology to their own ends: facilitating niche communications, preserving
dying languages, decentralizing currency controls, and other technologies that
efface the traditional boundaries between ordinary people and social power. So
that the computer is not just a technological artifact, but become a social
institution whose analysis is wrapped up with the modern cultural dialogue.

Naming practices are one facet of culture whose modern destiny is tied up with the
Computing Age. By "naming practices", we mean the patterns and expectations
within a culture that govern the use of linguistic tokens for the
identification of self and others. The form and usage of names exhibits vast
variation over time and space; the discrete, database-esque demands of computers
on names requires a uniformity that undercuts the diverse realities that exist
between cultures.

In this paper, we will examine the cultural diversity of naming and its
sociological function. Then we will review cases in which people's names did
not "play nice" with an organisation's computer system, and see how the case was
handled in favor of the individual or the organisation. Finally, we will
evaluate strategies and efforts for ensuring that computer systems can handle as
broad a spectrum of the world's names as possible.
