\section{Introduction}

\begin{aquote}{Janice "Lokelani" Keihanaikukauakahihuliheekahaunaele}
You see, to some people in the world, your name is everything. If I say my name
to an elder Hawaiian (kupuna), they know everything about my husband's family
going back many generations...just from the name.
\end{aquote}

\subsection{Names}

Article 7 of the Convention on the Rights of the Child enshrines a person's
inalienable right to a name: "The child shall be registered immediately after
birth and shall have the right from birth to a name." \parencite{crc} A name is
the fundamental token which identifies a person \textit{qua} individual; it
transmits information about its bearer, including cultural idenitification,
genealogical history, religious beliefs, and linguistic origins.

At the same time, a name cannot remain the sole province of its owner; it must
facilitate interaction with the wider world as a means of address. A name is
\textit{a priori} useless if others people in the environs cannot pronounce it,
write it, or remember it. Names thus serve functions on a spectrum ranging from
the expressive to the utilitarian.

\subsection{Computers and Writing Systems}

The "digital age" has changed writing from a free and individualized practice to
a one based in a discrete and logical structure consisting of discrete glyphs.
Since most early development of computers took place in the United States,
English gained a natural ascendancy over other languages in the field of digital
communication. English, perhaps as a coincidence, is also one of the easiest
languages to represent in code, requiring at the bare minimum just the 26
non-accented characters of the English alphabet, perhaps with some punctuation
and numbers. As computer technology spread far beyond the labs of Silicon
Valley, there were some associated growing pains with the need to encode a vast
array of linguistic forms, including diacritical marks, right-to-left scripts,
and pictographic writing systems.
