\section{Introduction}

In this paper, we will explore how digital information storage is transforming
personal names; naming practices are a facet of culture tied up with the
Computing Age. By naming practices, we mean patterns and norms of a culture that
govern use of linguistic tokens for identifying self and others. The forms of
personal names and customs by which they are acquired vary immensely across time
and space; yet in a world powered by SQL queries, a fluid, variable, or
context-dependent naming system is anathema to the desire of governments and
organizations to accurately identify individuals on a massive scale. In other
words, the importance of digital records to the modern functioning of powerful
institutions is enforcing the norm that whatever name you choose, it must be
compatible with the computer infrastructure used by the insitutions. Although
the ability of computers to represent a wide (linguistic) array of textual
content has improved greatly in recent years with standards such as Unicode, it
is still incomplete.  Large amounts of software rely on legacy standards (ASCII,
for example, especially in USA) and if there's no pressing need to update, often
the funds are not spent to improve representation. Other factors that may
inhibit a name's use with computers is excessive length, symbolic content, or
context dependence.

How are people affected when their name doesn't "play nice" with the computer?
We will look at several cases of people whose names could not be properly
represented in digital form because of length, character limitations, etc and
how they are denied equal access to government documents and/or services. How do
the persons feel about possibly changing their names for the convenience of
administrative technology? How willing are governments or companies to rectify
the situation? These are issues at the crux of the modern discourse on how to
balance administrative efficiency with respect for human dignity; in short, how
to stave off alienation in a digital world.

First, we'll review the worldwide diversity of naming customs and consider what
sociological functions are played by a name's contents. Then we will examine the
impact of computing on personal names. This entails examining some "bizarre
news" cases, in which people's names do not "play nice" with a computer system,
but also prompts more serious questions, such as how digital institutions are
reshaping personal and cultural identities, and whether persons' names in
developing countries are adequately supported by current digital infrastructure.
Finally, we evaluate strategies and efforts to ensure that computer systems can
support as broad a spectrum of the world's names as possible and that new
technologies will preserve and protect a person's unique identities. We will
consider each of the following research questions:

\begin{itemize}
\item How do computers influence the form that personal names may take?
\item How can computers impact personal and cultural identities?
\item How can we protect the social content of names in developing countries
and/or among minority-language speakers?
\end{itemize}
