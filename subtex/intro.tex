\section{Introduction}

This paper examines naming practices in several countries from a sociolinguistic
perspective. Although naming is universally held to be the prerogative of a
child's parents \parencite{alford87}, most states abridge this right to some
extent. In the case of "bizarre" names like "Adolf Hitler" and "Ghoul Nipple"
\parencite{larson11}, overruling parental choice is a clear-cut measure to
protect child welfare. However, many policies are instituted for more subtle
aims, including bureaucratic efficiency and linguistic policy. We will compare
naming laws in four jurisdictions, namely California, Lithuania, Iceland, and
the People's Republic of China. We observe that the rationale and methods for
restricting names are influenced by the country's official language policies and
status of the languages themselves.

Historical and ethnographic evidence suggests naming was traditionally a
flexible process, in which people used multiple names depending on context or
stage of life \parencite{alford87} \parencite{scott02}. Bureaucrats' need for
"synoptic legibility" drove the levelling of onomastic form under centralised
governments \parencite{scott02}. Today computer technology allows institutions
to efficiently process data, while limiting the possible forms and thus personal
expression available through names. The capabilities of computer systems are
crucial to the naming laws discussed in China and California. On the other hand,
names can be used as a tool to promote a specific language within a country, as
part of a comprehensive language policy. A relatively well-known example is
Iceland's list of approved names, which must accord with the Icelandic language.
Many countries (Lithuania) have similar regulations to Iceland's which seek to
impose a particular linguistic form to names within the country. We will find a
distinction between the policies of large languages (English and Chinese), which
are able to set the administrative agenda without considering the effect on
minority languages, and small national languages (Icelandic and Lithuanian),
which adopt language and naming policies in reaction to the looming threat of
global English.

Even through the legal and linguistic details, however, we should keep our
interest focused on the individual human rights issues involved. How do citizens
feel about changing their names for the convenience of administration? How
flexible or willing are governments or companies to rectify the situation? These
are issues at the crux of the modern discourse on how to balance administrative
efficiency with respect for human dignity; in short, how to stave off alienation
amidst the bureaucracy.

First we frame the discussion by considering the multifaceted forms and social
rôles played by personal names worldwide, as well as efforts by centralised
states to streamline names since the Early Modern period. Then we examine the
specific name policy of each case study, drawing appropriate comparisons where
necessary. Finally, we call for governments to adopt updated Unicode standards
to accept as wide a range of name formats as feasible. The importance of
recognising and valuing diverse name practices will only grow in the global
society.

\subsection{Research Questions}

\begin{itemize}
\item How do governments or institutions limit the choice of personal names?
\item How can governments accommodate a wider range of languages and formats for
	names?
\end{itemize}
