\section{Introduction}

This paper examines naming practices in several naming policies in several
countries from a sociolinguistic perspective. Although naming is universally
held to be the prerogative of a child's parents \parencite{alford88}, most
states abridge this right to some extent. In the case of "bizarre" names like
"Adolf Hitler" and "Ghoul Nipple" \parencite{larson09}, overruling parental
choice is a clear-cut decision for child welfare. However, many policies are
instituted for more subtle aims, including bureaucratic efficiency and
linguistic policy. We will comapre naming laws in four jurisdictions, namely
California, Lithuania, Iceland, and the People's Republic of China. We observe
that the rationale and methods for restricting names are influenced by the
country's official langage policies and status of the languages themselves.

Historical and ethnographic evidence suggests naming was traditionally a
flexible process, in which people used multiple names depending on context or
stage of life \parencite{alford88} \parencite{scott02}. As \textcite{scott02}
notes, bureaucrats' need for "synoptic legibility" has driven the levelling of
onamastic form under centralised governments. For instance, the use of
patrilineal surnames began in England and Turkey during centralising régimes
like Norman England and Atatürk's government, respectively. Today the effect is
heightened with computer technology, which allows institutions to efficiently
process data, while limiting the possible forms and thus personal expression
available through names. On the other hand, names can be used as a tool to
promote a specific language within a country, as part of a comprehensive
language policy. A relatively well-known example is Iceland's list of approved
names, which must accord with the Icelandic language. Many countries have
similar regulations to Iceland's which seek to impose a particular linguistic
form to names within the country.

In these cases, we must consider the people whose names either could not be
properly represented in a digital format or did not accord with the government's
naming policies. How do the persons feel about changing their names for the
convenience of administration? How flexible or willing are governments or
companies to rectify the situation? These are issues at the crux of the modern
discourse on how to balance administrative efficiency with respect for human
dignity; in short, how to stave off alienation in a modern world.

First we frame the discussion by considering the multifaceted forms and social
rôles played by personal names worldwide, as well as efforts by centralised
states to streamline names since the Early Modern period. Then we examine the
specific name policy of each case study, drawing appropriate comparisons where
necessary. Finally, we call for governments to adopt updated Unicode standards
to accept as wide a range of name formats as feasible. The importance of
recognising and valuing diverse name practices will only grow in our globalised
age. We will consider each of the following research questions:

\begin{itemize}
\item How do governments or institutions limit the choice of personal names?
\item How can governments accommodate a wider range of languages and formats for
	names?
\end{itemize}
