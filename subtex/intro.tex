\section{Introduction}

Modern governments wield considerable power in the private affairs of citizens.
This paper explore restrictions persons face when naming their children or
adopting new names for themselves. The naming process is cross-culturally
regarded as the prerogative of the child's parents or close relatives
\parencite{alford02}. Yet the right is not absolute and many state laws
circumscribe naming possibilities to a some extent. As \textcite{scott02} notes,
the administrative desire for "synoptic legibility" is a driving force changing
the form of personal names in industrialised societies, eg. through the
standardisation of surnames, approved names lists, or official language laws. On
the other hand, human rights specialists increasingly recognise the importance
of names as a cultural vector, calling for reform to allow linguistic expression
of names in an official context \parencite{varennes15}.

We will make a distinction between policies that are founded primarily in
administrative convenience and those that are part of a more comprehensive state
language policy. In many cases, the restrictions are issued to promote
administrative convenience or compatibility with digital systems, which require
that names adhere to a particular form. In other cases, however, the government
takes a more active role in name administration, such as by promulgating an
approved names list (Iceland) or by subjecting them to a process of
administrative overrule (Québec), thereby promoting a social (or ethnocentric)
policy agenda.

On a personal level, how are people affected when their name doesn't "play nice"
with the system? We will look at several cases of people whose names either
could not be properly represented in a digital format or did not accord with the
government's naming policies. How do the persons feel about changing their names
for the convenience of administration? How flexible or willing are governments
or companies to rectify the situation? These are issues at the crux of the
modern discourse on how to balance administrative efficiency with respect for
human dignity; in short, how to stave off alienation in a modern world.

First, we will review the worldwide diversity of naming customs and consider
what social functions are played by a name's contents. Then we will examine the
restrictions on personal naming. Then we will examine cases in which people's
names do not "play nice" with a computer system, or are rejected by an
administrator. These stories typically come from the "bizarre news" section of a
newspaper, but they prompt more serious questions, such as how concentrated
power is reshaping personal and cultural identities. These cases typify either a
need for administrative convenience, or an social naming policy. In some cases
we find a combination of the two rationales. We will consider each of the
following research questions:

\begin{itemize}
\item How does institutional power limit the choice of personal names?
\item How can computers impact personal and cultural identities?
\item How can we protect the social content of names in developing countries
and/or among minority-language speakers?
\end{itemize}
