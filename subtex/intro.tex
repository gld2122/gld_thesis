\section{Introduction}

In our "digital age" writing has transitioned from a free-flowing calligraphic
artform to a discretized logical structure consisting of well-defined glyphs.
The fact that most of the early development of computers took place in the
United States gave English an ascendancy over other languages for the purposes
of digital communication. English, perhaps as a coincidence, is also one of the
easiest languages to represent in code, requiring at the bare minimum just the
26 non-accented characters of the English alphabet, and perhaps some punctuation
and numbers mixed in. As computer technology has spread far beyond the labs of
Silicon Valley, there have been some growing pains associated with the need to
include a variety of linguistic practices in digital form, including diacritical
marks, right-to-left scripts, pictographic writing systems, and more.
