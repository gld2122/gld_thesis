\section{Introduction}

\begin{aquote}{Janice "Lokelani" Keihanaikukauakahihuliheekahaunaele}
You see, to some people in the world, your name is everything. If I say my name
to an elder Hawaiian (kupuna), they know everything about my husband's family
going back many generations...just from the name.
\end{aquote}

\subsection{Names}

Article 7 of the Convention on the Rights of the Child enshrines a person's
inalienable right to a name: "The child shall be registered immediately after
birth and shall have the right from birth to a name." \parencite{crc}

\subsubsection{Intrinsic Functions}

A name is the fundamental token which identifies a person \textit{qua}
individual; thus it contains biographical information about its bearer. This
can including cultural , genealogical , religious , and linguistic. 

\begin{itemize}
\item \textit{Genealogical}: The world's major naming practices combine a
given name, which singles out particular person, with a family name or surname,
which identifies that individual as belonging to some family or clan.
\end{itemize}

We can call
this the \textit{intrinsic function} of a name.

\subsubsection{Extrinsic Functions}

At the same time, a name cannot remain the sole province of its owner or family;
it must facilitate interaction with the wider world as a means of address and
identification. If a name affirms your status as an individual, it no less
affirms your status as a citizen of your country, resident of your city,
customer of your electric service provider, holder of your credit card, employee
of your company, and recipient of your parking ticket. A name is worth nothing
if others people in the environs cannot pronounce it, write it, or remember it.
Names thus serve functions on a spectrum ranging from the expressive to the
utilitarian. We can call this the \textit{extrinsic function} of a name.

\subsection{Computers}

In the "digital age" has changed writing from a free and individualized practice
to a one based in a discrete and logical structure consisting of discrete
glyphs. Since most early development of computers took place in the United
States, English gained a natural ascendancy over other languages in the field of
digital communication. English, perhaps as a coincidence, is also one of the
easiest languages to represent in code, requiring at the bare minimum just the
26 non-accented characters of the English alphabet, perhaps with some
punctuation and numbers. The ASCII standard encoding, with 127 available code
points, is more than enough to represent the English language in digital form.
Thus, organizations which deal only infrequently with non-English text have been
slow to update their databases to Unicode standards.
