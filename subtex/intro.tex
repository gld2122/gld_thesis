\section{Introduction}

\begin{aquote}{Janice "Lokelani" Keihanaikukauakahihuliheekahaunaele}
You see, to some people in the world, your name is everything. If I say my name
to an elder Hawaiian (kupuna), they know everything about my husband's family
going back many generations...just from the name.
\end{aquote}

\subsection{Names}

Article 7 of the Convention on the Rights of the Child enshrines a person's
inalienable right to a name: "The child shall be registered immediately after
birth and shall have the right from birth to a name." \parencite{crc}

\subsubsection{Functions of a Name}

A \textit{personal name} is the canonical token by which we single out some
person as an individual; we can call this person the \textit{referent} of the
name. Through the name we can locate the person in our biographical index,
thereby accessing all the full entry: nationality, personality, relationships,
family, etc. This function of a name is rather neutral regarding content; any
token which is reasonably unique will allow us to navigate this index. There is
no reason that "John" is a more effective identifier than "12345678".

What interests us more here is the role played by the content of the name and
the complex of information it carries, be it genealogical, cultural, linguistic,
religious, etc. These functions sit on a spectrum between individualizing
functions and connecting functions, or what Elias calls "I-identity" and
"We-identity". \parencite{elias91} \parencite[711]{finch08}

Individualizing functions are those aspects of a name which make a statement on
the individual themselves. The clearest manifestation of this is the choice of a
child's forename; indeed this makes more of a statment on the parent than the
child itself. "In selecting a name (especially for a first-born child) parents
are not only determining the personhood of their child but are also taking a key
step in defining their own new identity as parents." \parencite[718]{finch08}
Hence a parent can name their child something "beautiful" like "Isabella" or
something "strong" like "Samson".

Connecting functions are facets that locate the individual within a larger
milieu. Most basically this takes the form of surnames, which in Anglophone
societies identify the paternal family unit to which the individual belongs. We
can find more subtle connecting functions, however. Choosing a first name after
an older ancestor connects you to a more specific family relationship. And even
the linguistic or religious connotations carried within the first or last name
can connect a person to or set them apart from the dominant society in which
they live.

\begin{itemize}

\item \textit{Genealogical}: 

\item \textit{Religious}: For religious persons, faith system is 

\end{itemize}

We can call
this the \textit{intrinsic function} of a name.

\subsubsection{Connecting Functions}

At the same time, a name cannot remain the sole province of its owner or family;
it must facilitate interaction with the wider world as a means of address and
identification. If a name affirms your status as an individual, it no less
affirms your status as a citizen of your country, resident of your city,
customer of your electric service provider, holder of your credit card, employee
of your company, and recipient of your parking ticket. A name is worth nothing
if others people in the environs cannot pronounce it, write it, or remember it.
Names thus serve functions on a spectrum ranging from the expressive to the
utilitarian. We can call this the \textit{extrinsic function} of a name.

\subsection{Computers}

In the "digital age" has changed writing from a free and individualized practice
to a one based in a discrete and logical structure consisting of discrete
glyphs. Since most early development of computers took place in the United
States, English gained a natural ascendancy over other languages in the field of
digital communication. English, perhaps as a coincidence, is also one of the
easiest languages to represent in code, requiring at the bare minimum just the
26 non-accented characters of the English alphabet, perhaps with some
punctuation and numbers. The ASCII standard encoding, with 127 available code
points, is more than enough to represent the English language in digital form.
Thus, organizations which deal only infrequently with non-English text have been
slow to update their databases to Unicode standards.
