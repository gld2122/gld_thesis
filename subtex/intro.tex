\section{Introduction}

The bureaucratic state plays a considerable rôle in citizens' private affairs,
including their names. This paper explores legal restrictions on the naming of
children or adoption of new names by adults. Selecting names is universally
considered the prerogative of a child's parents or close relatives
\parencite{alford88}, yet many state laws circumscribe naming possibilities.
While some names are prohibited for posing a threat to the child's welfare
("Adolf Hitler"), this paper focuses on restrictions caused by the name's
linguistic form. We observe that two common rationale for naming laws are to
promote administrative convenience and integrity of government information on
one hand, and to promote desired social, cultural, or linguistic outcomes on the
other.

Historical and ethnographic evidence suggests naming was traditionally a
flexible process, in which people used multiple names depending on context or
stage of life \parencite{alford88} \parencite{scott02}. As \textcite{scott02}
notes, bureaucrats' need for "synoptic legibility" has driven the levelling of
onamastic form under centralised governments. For instance, the use of
patrilineal surnames began in England and Turkey during centralising régimes
like Norman England and Atatürk's government, respectively. Today the effect is
heightened with computer technology, which allows institutions to efficiently
process data, while limiting the possible forms and thus personal expression
available through names.

On the other hand, names can be used as a tool to promote a specific language
within a country, as part of a comprehensive language policy. A relatively
well-known example is Iceland's list of approved names, which must accord with
the Icelandic language. Many countries have similar regulations to Iceland's
which seek to impose a particular linguistic form to names within the country.
In this review, case studies from California, Iceland, Lithuania, and China
illustrate how each of these motives interact to produce a government naming
policy.

In these cases, we must consider the people whose names either could not be
properly represented in a digital format or did not accord with the government's
naming policies. How do the persons feel about changing their names for the
convenience of administration? How flexible or willing are governments or
companies to rectify the situation? These are issues at the crux of the modern
discourse on how to balance administrative efficiency with respect for human
dignity; in short, how to stave off alienation in a modern world.

First we review the diversity in and social meaning of anthroponyms. Then we
examine restrictions on personal naming in California, Iceland, Lithuania, and
China. These cases typify either a need for administrative convenience, or an
social naming policy. In some cases we find a combination of the two rationales.
They also prompt serious questions, like how concentrated power has reshaping
personal and cultural identities. We will consider each of the following
research questions:

\begin{itemize}
\item How do governments or institutions limit the choice of personal names?
\item How can governments accommodate a wider range of languages and formats for
	names?
\end{itemize}
