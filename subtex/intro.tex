\section{Introduction}

The "bureaucratic" state plays a considerable rôle in citizens' private affairs,
including their names. This paper explores legal restrictions on the naming of
children or adoption of new names by adults. Selecting names is universally
considered the prerogative of a child's parents or close relatives
\parencite{alford88}. Yet this right is not absolute; many state laws
circumscribe naming possibilities. This paper argues that two common rationale
for naming laws are to promote administrative convenience and the integrity of
government information on one hand, and to promote desired social, cultural, or
linguistic outcomes on the other.

Historical and ethnographic evidence suggests naming was traditionally a
flexible process, in which people use by multiple names depending on context or
stage of life \parencite{alford88} \parencite{scott02}. As \textcite{scott02}
notes, bureaucrats' need for "synoptic legibility" has driven the levelling of
onamastic forms under centralised governments. For instance, the rise of
patrilineal surnames occurred within centralising régimes such as Norman England
and Republican Turkey. Today the effect is heightened by computer technology,
which allows institutions to efficiently process data, while limiting the
possible forms and thus personal expression available through names.

On the other hand, names can be used as a tool to promote a specific language
within a country, as part of a comprehensive language policy. A relatively
well-known example is Iceland's list of approved names, which must accord with
the Icelandic language. Many countries have similar regulations to Iceland's
which seek to impose a particular linguistic form to names within the country.
In this review, case studies from California, Iceland, Lithuania, and China will
illustrate how each of these motives interact to produce a government naming
policy.

In these cases we must consider the people whose names either could not be
properly represented in a digital format or did not accord with the government's
naming policies. How do the persons feel about changing their names for the
convenience of administration? How flexible or willing are governments or
companies to rectify the situation? These are issues at the crux of the modern
discourse on how to balance administrative efficiency with respect for human
dignity; in short, how to stave off alienation in a modern world.

First we review the syn- and diachronic diversity in and social meaning of
anthroponyms. Then we examine the restrictions on personal naming. they also
prompt serious questions, like how concentrated power has reshaping personal and
cultural identities. These cases typify either a need for administrative
convenience, or an social naming policy. In some cases we find a combination of
the two rationales. We will consider each of the following research questions:

\begin{itemize}
\item How does institutional power limit the choice of personal names?
\item How can computers impact personal and cultural identities?
\item How can we protect the social content of names in developing countries
and/or among minority-language speakers?
\end{itemize}
