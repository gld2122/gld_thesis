\section{Introduction}

\begin{aquote}{Janice "Lokelani" Keihanaikukauakahihuliheekahaunaele}
You see, to some people in the world, your name is everything. If I say my name
to an elder Hawaiian (kupuna), they know everything about my husband's family
going back many generations...just from the name.
\end{aquote}

\subsection{Names}

A name is a fundamental identifier of a person \textit{qua} individual; as such,
it seen as a highly personal issue, and the choice of name by the child's
parents may carry cultural, genealogical, religious, linguistic information. 

\subsection{Computers and Writing Systems}

The "digital age" has changed writing from a free and individualized practice to
a one based in a discrete and logical structure consisting of discrete glyphs.
Since most early development of computers took place in the United States,
English gained a natural ascendancy over other languages in the field of digital
communication. English, perhaps as a coincidence, is also one of the easiest
languages to represent in code, requiring at the bare minimum just the 26
non-accented characters of the English alphabet, perhaps with some punctuation
and numbers. As computer technology spread far beyond the labs of Silicon
Valley, there were some associated growing pains with the need to encode a vast
array of linguistic forms, including diacritical marks, right-to-left scripts,
and pictographic writing systems.
