\section{Introduction}

\subsection{The Computer Age}

Since the 1960s, computers have become central to financial, governance, and
social infrastructures. This makes the computer essential to the functioning of
modern society; its impact is felt across all endeavours and all regions of the
world. Hence computing sits at the crossroads of pressing social discourses,
including centralisation, globalisation, alienation, etc. While it would be
simplistic to attribute broad processes solely to a change in technology (the
centralisation of state power, for instance, is ongoing since at least the High
Middle Ages in the case of Western Europe) the computer has accelerated the
spread of such phenomena.

Still, the victims of centralisation and globalisation find ways to harness
technology to their own ends: facilitating niche communications, preserving
dying languages, decentralizing currency controls, and other technologies that
efface the traditional boundaries between ordinary people and power structures.
So that the computer is not just a technological artifact, but become an
institution whose analysis is intertwined with the cultural dialogue.

\subsection{Plan}

Naming practices are one facet of culture whose modern destiny is tied up
with the Computing Age. By "naming practices", we mean the patterns and
expectations within a culture that govern the use of linguistic tokens for the
identification of self and others. The form and usage of names exhibits vast
variation over time and space; however, the discrete and automated operation of
computer databases demands a uniformity in name which undercuts the diverse
reality that exists between cultures.

In this paper, we first review the worldwide diversity of naming customs and
their sociological functions. Then we examine the impact of computing on
personal names. This entails both "bizarre news" cases, in which people's names
do not "play nice" with a computer system, but also more serious questions such
as how computers are reshaping personal and cultural identities, and persons'
names in developing countries are adequately supported by dominant computer
infrastructure. Finally, we evaluate strategies and efforts to ensure that
computer systems can support as broad a spectrum of the world's names as
possible and that new technologies will preserve and protect person's unique
identities.
