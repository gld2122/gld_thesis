\section{Introduction}

This paper examines naming practices and policy in several countries from a
sociolinguistic perspective. Although naming is held to be the prerogative of a
child's parents \parencite{alford87}, most states abridge the right to some
extent. In the case of "bizarre" names like "Adolf Hitler" and "Ghoul Nipple"
\parencite{larson11}, overruling parental choice is a clear-cut measure to
protect child welfare. But many policies are instituted for other aims, such as
increasing bureaucratic efficiency and supporting linguistic ideologies. We
will compare the naming laws of four jurisdictions: the prohibition of
diacritics on birth certificates in California, case and spelling assimilation
in Latvia, the Personal Names List Iceland, and character restrictions in
People's Republic of China. We observe that the methods and justifications for
restricting names are influenced by the country's official language policies,
and individual's names are frequently used as a venue for implementing or
preventing linguistic changes.

History and ethnography suggest that naming was traditionally a flexible
process, as people used multiple names depending on context or their stage in
life \parencite{alford87} \parencite{scott02}. Bureaucrats' need for "synoptic
legibility" drove the levelling of onomastic form under centralising
governments \parencite{scott02}. Computerised data processing by governments
has increased the process in the last half century. The impact of computers
plays a major role in the regulations we discuss from California and China.

On the other hand, names can be used as a tool to promote a specific language
within a country, as part of a comprehensive language policy. A well-known
example is Iceland's Personal Names Committee, which determines whether a name
is compatible with the Icelandic language. Other countries like Latvia have
similar regulations, which seek to prop up the small national language by
mandating its use in names and other areas of society.

We will find a distinction between the policies of large languages (English and
Chinese), which are able to set the administrative agenda without considering
the effect on minority languages, and small national languages (Icelandic and
Lithuanian), which bolster national languages' stability in the post-colonial
era and defend against the looming threat of global English.

The regulatory and linguistic details, however, are only a backdrop for the
individual human rights issues involved. Even as states pursue their linguistic
interests, a rich human rights tradition defends the right to privacy against
government interference. How do citizens feel about changing their names for
administrative convenience or linguistic agendas? These are issues at the crux
of the modern discourse on how to balance administrative efficiency with
respect for human dignity.

This study contributes a comparative analysis of naming laws and their
relations to language policy and language rights. We present background
regarding the social and administrative functions of names, as well as the
legal view on naming practices in common law and human rights, which prepares
us for our individual review of case studies.  We then consider four case
studies from California, China, Iceland, and Latvia. The comparative lens
illuminates the fact that personal names have been politicised and used by
governments for language management. Finally, we conclude by considering the
human rights implications of naming restrictions, which call for the moderation
of linguistic policymaking with respect for privacy and self-determination.
