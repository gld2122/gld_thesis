\section{Introduction}

¿Cómo se llama? This first question any language student must ask may not have
such an easy answer. Names can carry a whole array of social baggage (heritage),
be it religious, cultural, linguistic, national, etc. My own name Gaberiel is a
good (misspelled) Biblical name with Romantic flair. Never mind that I am
neither religious nor particularly Romantic. The semantic relation between a
name and its bearer is not one-to-one; but then, what assumptions do we make
about someone named Wang Xiaoqin?

Asking the cómo-se-llama in Western context presupposes a fixed answer. Imagine
the IRS' dismay if Walter demanded that he should addressed as Jean-Pierre for
the entirety of 2020. (It is, after all, a leap year.) The IRS would cease to
function if it had to honor such naming practices for 320 million citizens.
Administrative convenience, as \textcite{scott02} finds, has been the primary
factor promoting the spread of fixed legal names and hereditary surnames. For
example, in England, surnames were first adopted by the landed Norman élite. As
the bureaucracy strengthened and middling types aspired to emulate their status.
The practice spread down the social ladder and by the end of the 18th century
had reached all parts of Great Britain. An accelerated programme as such
occurred in Turkey starting in 1934 under the Westernising régime of Atatürk.

As we can see, fixed legal names are not a fact of life; they are a
construction. In many non-industrialised societies, we find that naming is fluid
and context-dependent. For example, the Giriama…

Fluid naming practices presuppose relative familiarity and the sharing of social
context. Such practices cannot withstand a growing bureaucratic presence;
governments need a "synoptic" view of their populations \parencite{scott02}.

Today, computers are making the synoptic view ever more crystal clear for
decision makers, providing precise and updated information from all parts of the
Empire. On top of the requirement of a fixed legal name, it is now expected (if
not mandated) that the name be compatible with institutional systems of digital
record-keeping. This is the name that will go on your birth certificate, your
passport, your driver's licence, your social security card; it is the name that
makes you You.

How are people affected when their name doesn't "play nice" with the computer?
We will look at several cases of people whose names could not be properly
represented in digital form because of length, character limitations, etc. What
are the effects of such a predicament? How do the persons feel about possibly
changing their names for the convenience of administrative technology? How
willing are governments or companies to rectify the situation? These are issues
at the crux of the modern discourse on how to balance administrative efficiency
with respect for human dignity; in short, how to stave off alienation in a
digital world.

In this paper, we will explore how the computers are transforming the names we
call ourselves and each other; naming practices are a facet of culture whose
modern destiny is tied up with the Computing Age. By "naming practices", we mean
the patterns and expectations within a culture that govern the use of linguistic
tokens for the identification of self and others. The forms and customs of
naming exhibit vast variation across time and space; yet in a world powered by
SQL queries, a fluid, context-dependent naming system is anathema to the desire
of governments and organizations to accurately track individual persons on a
massive scale.

First, we'll review the worldwide diversity of naming customs and consider what
sociological functions are played by a name's contents. Then we will examine the
impact of computing on personal names. This entails both "bizarre news" cases,
in which people's names do not "play nice" with a computer system, but also more
serious questions, such as how computers are reshaping personal and cultural
identities, and whether persons' names in developing countries are adequately
supported by dominant computer infrastructure. Finally, we evaluate strategies
and efforts to ensure that computer systems can support as broad a spectrum of
the world's names as possible and that new technologies will preserve and
protect a person's unique identities. We will consider each of the following
research questions:

\begin{itemize}
\item How do computers influence the form that personal names may take?
\item How can computers impact personal and cultural identities?
\item How can we protect the social content of names in developing countries and
among minority-language speakers?
\end{itemize}
