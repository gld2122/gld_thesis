\section{Introduction}

This paper examines naming practices and policy in several countries from a
sociolinguistic perspective. Although naming is held to be the prerogative of
parents \parencite{alford87}, most states abridge the right to some extent. In
case of "bizarre" names like "Adolf Hitler" and "Ghoul Nipple"
\parencite{larson11}, overruling parental choice is a clear-cut measure
protecting child welfare. Other policies aim at controlling the content of
names to increase bureaucratic efficiency, implement linguistic policies, or
support other state goals. We will compare the naming laws interaction with
linguistic policy in four jurisdictions: the prohibition of diacritics on birth
certificates in California, case and spelling assimilation in Latvia, the
Personal Names List Iceland, and character restrictions in People's Republic of
China. We observe that the methods and justifications for restricting names are
influenced by the country's official language policies and that individual's
names are frequently used as a venue for implementing or preventing linguistic
changes.

History and ethnography suggest that naming was traditionally a flexible
process, as people used multiple names depending on context or their stage in
life \parencite{alford87} \parencite{scott02}. Bureaucrats' need for "synoptic
legibility" drove the levelling of onomastic form under centralising
governments \parencite{scott02}. Computerised data management by governments
has accelerated this process in the last half century. One variety of naming
policies we consider, exemplified by cases in California and China, center on
the state collection of standardised demographic and identification information
(and obliquely, the impact of computers in maintaining this information).
Modern data processing implicitly limits the variety of names that may be
accepted.

On the other hand, names can be actively used as tools to promote a specific
language within a country, as part of a comprehensive language policy. A
well-known example is Iceland's Personal Names Committee, which determines
whether a name is compatible with the Icelandic language. Other countries like
Latvia have similar regulations, which seek to prop up the small national
language by mandating its use in names and other areas of society.

Thus we find a distinction between the policies of countries with strong (de
facto) official languages like English and Chinese, for which language-based
naming restrictions are primarily incidental and based on a strong state
apparatus. Countries with small national languages like Icelandic and Latvian,
on the other hand, aim to bolster their languages' stability in the
post-colonial era and defend against the looming threat of global English.

Naming restrictions raise human rights concerns, even if they are designed to
support a legitimate state interest in the official language. The cases we
consider herein center on the right to privacy recognised in the Due Process
Clause of the Fourteenth Amendment (in US law) and privacy clauses in
international treaties. Given the personal character of naming, citizens may
feel burdened by being forced to adopt a new name or conversely, being denied
the chance to change their name, for the sake of administrative convenience or
linguistic agendas? The question is thus how to balance administrative efficiency
with respect for human dignity.

This study contributes a comparative analysis of naming laws and their
relations to language policy and language rights. We present background
regarding the social and administrative functions of names, as well as the
legal view on naming practices in common law and human rights, preparing us for
the review of individual case studies. We then consider four cases from
California, China, Iceland, and Latvia. The comparative lens illuminates the
fact that personal names have been politicised and are used by governments for
language management. Finally, we conclude by considering the human rights
implications of naming restrictions, which call for the moderation of
linguistic policymaking with respect for privacy and self-determination.
