\section{Introduction}

Since the 1960s, computers have taken centre-stage in ensuring the smooth
functioning of society, by enabling high-efficiency storage and processing of
data. We feel this impact across all endeavours and all regions of the world.
Computing sits at the crossroads of pressing social discourses, such as
centralisation, globalisation, alienation, etc. Clearly it would be simplistic
to attribute these broad processes solely to a change in technology; the
centralisation of state power, for instance, has been ongoing since at least the
High Middle Ages in the case of Western Europe. Nonetheless, the computer acts
in the name of ruthless efficiency, accelerating the spread of such phenomena.

Still, the victims of centralisation and globalisation find ways to harness
technology to their own ends: facilitating niche communications, preserving
dying languages, decentralizing currency controls, and other technologies that
efface the traditional boundaries between ordinary people and power structures.
So that the computer is not just a technological artifact, but become an
institution whose analysis is intertwined with the cultural dialogue.

Naming practices are just one facet of culture whose modern destiny is tied up
with the Computing Age. By "naming practices", we mean the patterns and
expectations within a culture that govern the use of linguistic tokens for the
identification of self and others. The form and usage of names exhibits a vast
variation over time and space; however, the discrete and automated operation of
computer databases demands a uniformity in name which undercuts the diverse
reality that exists between cultures.

In this paper, we will review the cultural diversity of naming and its
sociological function. Then we will review the impact of computing on personal
names. This will entail some "bizarre news" cases, in which people's names did
not "play nice" with a computer system, but also more serious perspectives of
how computers are reshaping our personal and cultural identities. Finally, we
will evaluate strategies and efforts for ensuring that computer systems can
handle as broad a spectrum of the world's names as possible and that new
technologies will preserve and protect person's unique identities.
