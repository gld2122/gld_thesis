\section{Passports}

The standard protocol for machine-readable travel documents is Document 9309
issued by the International Civil Aviation Organization. Since all countries
must operate on shared standard, it is necessary to forge a compromise between
cultural openness and international security. The ICAO-9309 standard gives
significant flexibility to the inclusion of national characters.

The passport data page is divided into two sections: the Visual Inspection Zone
(VIZ) and the Machine Readable Zone (MRZ). Countries may fill the required
fields of the VIZ however they desire in the national language, provided a
transcription or translation is also provided into English, Spanish, or French.
Thus compliant passports do not per se coax a country toward the adoption of
standard alphabets while still allowing international cooperation.

In the Machine Readable Zone, a transcription into the organization's approved
ASCII subset (0-9, A-Z, <) is required, for the purpose of machine recognition.
For Latin alphabet languages, most characters containing diacritics simply have
the mark dropped, although some characters have recommended control sequences to
losslessly transliterate the character. The document spells out a more extensive
scheme for Cyrilic and Arabic characters which allows nearly lossless recovery
of the original form from the highly schematic MRZ specification. There is even
a sample Python program for converting from the MRZ to Unicode Arabic.

\subsection{Arabic}

For example, the Arabic name ابو بكر محمد بن زكريا الرازي would be rendered in the MRZ:

ABW<BKR<MXHMD<BN<ZKRYA<ALRAZY

While this looks almost incomprehensible when read by a human, the use of X as
an escape character allows a one-to-one transliteration back into the original
script. More examples below:

\includegraphics{../Articles/9309.3-appendix-b.5.9.png}

\subsection{Latin}

In the case of Roman script languages a distinction can be made regarding the
salience of diacritical marks. In the ICAO's recommendation, diacritics such as
the acute or grave accents over vowels are simply eliminated in the MRZ.
However, they provide methods of encoding more salient characters such as the
German umlaut vowels (ä,ö,ü) or the Spanish tilde ñ.

So the name "Térèsa Cañón" would become CANXXON<<TERESA in the MRZ. Likewise,
"Wilhelm Furtwängler" would become FURTWAENGLER<<WILHELM. (b.4.2)
