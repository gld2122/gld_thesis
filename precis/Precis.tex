\documentclass{article}

\usepackage[margin=1in]{geometry}
\usepackage{fancyhdr}
\usepackage{comment}
\usepackage{fontspec}
\usepackage{defreitasgabe}
\usepackage[backend=biber]{biblatex}
\usepackage[english]{babel}
\usepackage[autostyle,english=american]{csquotes}

\pagestyle{fancy}
\fancyhf{}
\lhead{Précis: English as Official Language}
\rhead{Gabe DeFreitas (gld2122)}
\lfoot{\today}
\rfoot{\thepage}
\frenchspacing

\setmainfont{Georgia}

\DeclareNameAlias{author}{last-first}

\MakeOuterQuote{"}
\addbibresource{Bibliography.bib}

\begin{document}

\section*{Introduction}

Although the United States has no official language specified in the US
Constitution, one might assume that English holds this title, considering its
ubiquitous role in American public life. A majority of US states have indeed
solidified the official status of English through constitutional or statutory
means ("Official English" laws). American language law at the state-level will
be my primary object of study.

By conducting a comparative study on state language law, we will study how
governments influence the use of the English langauge in official and unofficial
settings, compare laws in their forms and effects, and identify reliable data to
guide us in this task. Comparisons to other countries may assist as well, for
example, the well-known "Loi Toubon" requiring the use of French by public
officials in France \parencite{Calvet96}.

Sociolinguistics affirms that language does not exist in a theoretical vacuum;
it carries social evaluations and entails real outcomes for real speakers of a
language variety. Through this perspective, we will analyse how the United
States' linguistic policy supports the dominant status of English and/or
protects the rights of minority-language speakers. We will examine how social
evaluation and lawmaking intermingled throughout American history and the
effects of this on language communities.

\section*{Consequence}

The percentage of Americans with a mother tongue other than English continues to
rise (17.89\% in 2000 to 19.89\% in 2010). Yet some states, like California,
with large numbers of minority-language speakers, also have Official English
laws on the books; the views and concerns of these minority-language speakers
must be considered vis-à-vis the majority language. Language issues often
reflect the struggle for political power between majority and minority in
democratic society.

The election of Donald Trump as President has sparked concern that public fears
of immigration are on the upswing. Distrust of foreigners is easily disguised as
concern over the purity of langauge. \textcite{Liu14} finds that national
salience of immigration fears (in national media and politics) is one of the
leading risk factors predicting the adoption of language legislation at the
state level. Direct democracy procedures facilitate the expression of this
"grassroots fears", and states with such provisions appear more likely to pass
Official English laws.

\section*{Deficit}

\textcite{Baron92} and other summaries of American language law give
comprehensive legislative overviews of the issue at federal and state levels. A
common concern with legislative analysis, however, is lack of followthrough.
Legislation often receives a lot of coverage at the time of its debate; once
implemented, it is difficult to isolate the effects and outcomes of the policy
and the "buzz" has moved on to new issues. Here we want to assess the outcomes
that state language initiatives have had for their citizens. This will be
accomplished by finding case studies and relevant statistics.

For example, in California, voters approved two "pro-English" referendums in
1986 and 1998, despite the state's massive Spanish-speaking population. This
state thus has one of the strongest Official English laws in the country.  Even
the naming of babies is affected: only the 26 letters of the English alphabet
are accepted on birth certificates, making it officially illegal to name a child
in California "José" or "María" \parencite{larson11}.

On the other end of the spectrum is a state like Illinois, where English is
established as the official language to the extent that the Cardinal is
established as the official bird, and the Monarch butterfly as its official
insect. One would expect this language policy to have less tangible effect than
California's.

\section*{Questions}

\begin{itemize}

\item Why did the US not adopt a federal language in the Constitution nor in the
  years that followed?
\item What factors predict the adoption of state language laws?
\item Have language laws succeeded in the ways their supporters hoped? And if so,
  how did they accomplish these goals?
\item What factors predict the outcomes of state language laws?
\item Why do direct democracy procedures increase the likelihood of linguistic
  regulation \parencite{Liu14}?
\item How do non-native English speakers responded to the Official English
  movement?
\item How can we read government discourse as an implicit statement on language
  status?

\end{itemize}

\begin{comment}

\section*{Political Linguistics}

\textit{Political linguistics} is a subfield of sociolinguistics that studies the ways
and extent by which governance structures influence or dictate the status and
form of language used within its jurisdiction. In this context, language becomes
a political resource over which actors can jockey for control. Once a political
choice regarding language has been made, implementation will take the form of
language planning, which can manipulate either the "corpus" or "status" of a
%language \parencite{Calvet96}. Such language planning measures will be designed
by trained linguists acting in a clinical, technocratic capacity, but one must
not forget the governmental power that lies behind the measures.


\section*{Language Policy in the United States}

\section*{State-by-State Results}

Comparing the contents and outcomes of state langauge laws will shed light on
what form effective political mandates will take.

\end{comment}

\nocite{*}
\printbibliography

\end{document}
