\documentclass{article}

\usepackage[margin=1in]{geometry}
\usepackage{comment}
\usepackage{fontspec}
\usepackage{defreitasgabe}
\usepackage[backend=biber]{biblatex}
\usepackage[english]{babel}
\usepackage[autostyle,english=american]{csquotes}

\MakeOuterQuote{"}

\addbibresource{Bibliography.bib}

\setmainfont{Georgia}

\begin{document}

\section*{Introduction}

While the United States has no "official" language specified by the US
Constitution, one could be forgiven for assuming that English holds this title,
considering its ubiquitous role in American public life. A majority of US
states, indeed, have codified the official status of English through
constitutional and/or statutory means. State language laws as such will be the
primary object of study in this project.

Sociolinguistics affirms that language does not exist in some theoretical
vacuum; it carries social evaluations and entails real outcomes for real
speakers of a particular language variety. Through a perspective of political
linguistics, we will deconstruct the linguistic policy of the United States to
understand how a multicultural nation ultimately abetted the hegemonic status of
English in the US.  We will look at how social evaluations and their legal
manifestations played out over the history of the United States, and the effects
(or lack thereof) of laws on American language communities.

I want to explicate the current state-level status of English as an official
language, classify the laws by their level of potency and intention, and search
for reliable data that can help us gauge the efficacy of specific state laws and
compare them to other measures taken in multilingual countries.

\section*{Political Linguistics}

Political linguistics is a subfield of sociolinguistics that studies the ways
and extent by which governance structures influence or dictate the status and
form of language used within its jurisdiction. In this context, language becomes
a political resource over which actors can jockey for control. Once a political
choice regarding language has been made, implementation will take the form of
language planning, which can manipulate either the "corpus" or "status" of a
language \parencite{Calvet96}. Such language planning measures will be designed
by trained linguists acting in a clinical, technocratic capacity, but one must
not forget the governmental power that lies behind the measures.

This is the framework in which we undertake this study. I hope to
contribute knowledge in two ways:

\begin{enumerate}
\item Examining the form and effects of the various state language laws in the
United States
\item Understanding how the United States became a monolingual political entity
in the context of a multicultural population, without the presence of a federal
language policy
\end{enumerate}

\section*{Language Policy in the United States}

\cite{Baron92} and similar summaries of US language law give a solid
"legislative" history of the issue at both federal and state levels. I have not
yet reviewed \parencite{Crawford94} but I believe it will be a good access point
to primary sources. Recent activity may be less well recorded. Another problem
is with follow through.  Legislation tends to have a lot of coverage at the time
of its debate directly prior to passage, when the issue is hot in the public
consciousness. Once implemented, however, it is difficult to isolate the effects
and outcomes of the policy, as the "buzz" has then moved on to new issues. This
is my fundamental goal herein: to analyse the outcomes state-level laws have
wrought for their citizens. This will be accomplished by the case studies
discussed below.

\section*{State-by-State Results}

One interesting case is California, where voters have approved two "pro-English"
referendums in 1986 and 1998, despite the state's massive Spanish-speaking
population. It has one of the strongest official English laws in the country.
This stance even affects the naming of babies: only the 26 letters of the
English alphabet are accepted on birth certificates, making it officially
illegal to name a child in California "José" or "María" \textcite{Larson11}.

The other end of the spectrum is a state like Illinois, where English is
established as the official language to the extent that the \textit{Cardinalis
cardinalis} is established as the official bird, and the Monarch butterfly as
its official insect. One would expect such a language policy to have little
concrete effect.

\printbibliography

\end{document}
