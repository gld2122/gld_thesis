\documentclass{article}

\usepackage[margin=1in]{geometry}
\usepackage{comment}
\usepackage{fontspec}
\usepackage{defreitasgabe}
\usepackage[backend=biber, style=authoryear-icomp]{biblatex}
\usepackage [english]{babel}
\usepackage [autostyle, english = american]{csquotes}

\MakeOuterQuote{"}

\addbibresource{Bibliography.bib}

\setmainfont{Georgia}

\begin{document}

\section*{Introduction}

Sociolinguistics affirms that language does not exist in some theoretical
vacuum; it carries social evaluations and entails real outcomes for real
speakers of a particular language variety. Through this perspective we will
deconstruct the linguistic régime of the United States to understand how such a
multicultural nation ultimately gave rise to the worldwide English monolith.
Thru the paradigm of political linguistics, we will look at how these social
evaluations and their juridical manifestations have played out over the history
of the United States, and the effects (or lack thereof) of laws on American
language communities.

Specifically, I want to cover the current state-level status of English as an
official language, grouping the laws by their general provisions, and search for
reliable data that can help us gauge the potency of particular state laws
(before and after enactment).

\section*{Political Linguistics}

Political linguistics studies the ways and extent to which "official" governance
structures can influence the use, status, or form of language among its
constituents. \cite{Calvet96} This is the framework in which we undertake this study. I hope to
contribute knowledge in two ways:

\begin{enumerate}
\item Examining the form and effects of the various state language laws in the
United States
\item Understanding how the United States became a monolingual political entity
in the context of a multicultural population, without the presence of a federal
language policy
\end{enumerate}

\section*{US Demography}

\section*{Language Policy in the United States}

\cite{Baron92} gives a thorough overview of language laws throughout the history
of the United States.

\section*{State-by-State Case Studies}

\section*{Comparative Analysis and Conclusions}

\printbibliography

\end{document}
