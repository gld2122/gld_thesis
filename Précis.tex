\documentclass{article}

\usepackage[margin=1in]{geometry}
\usepackage{comment}
\usepackage{fontspec}
\usepackage{defreitasgabe}
\usepackage[backend=biber, style=authoryear-icomp]{biblatex}
\usepackage [english]{babel}
\usepackage [autostyle, english = american]{csquotes}

\MakeOuterQuote{"}

\addbibresource{Bibliography.bib}

\setmainfont{Georgia}

\begin{document}

\section*{Introduction}

Sociolinguistics affirms that language does not exist in some theoretical
vacuum; it carries social evaluations and entails real outcomes for real
speakers of a particular language variety. Through this perspective we will
deconstruct the linguistic régime of the United States to understand how such a
multicultural nation ultimately gave rise to the worldwide English monolith.
Thru the paradigm of political linguistics, we will look at how these social
evaluations and their juridical manifestations have played out over the history
of the United States, the effects of various laws on American language
communities.

\section*{Political Linguistics}

% From Calvet

\section*{Deficit Mindset}

On the controversial issue of bilingual education, Huerta et al. writes that
"nos quiere hacer creer que es una práctica "neutral,"" and that "la dimensión
política, social, y de poder \ldots se deja fuera de la discusión."

% From Huerta

\section*{Language Repression}

\section*{Language in the United States}

% Summary of Baron

\section*{Bilingual Education}

\section*{Research Plan}

\begin{enumerate}
\item Historical and modern US demographics
\item History and literature review on Official English movement and state-level
language laws
\item International comparisons
\end{enumerate}

\end{document}
