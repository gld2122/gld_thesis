\documentclass{article}

\usepackage[margin=1in]{geometry}
\usepackage{comment}
\usepackage{fontspec}
\usepackage{defreitasgabe}
\usepackage[backend=biber, style=authoryear-icomp]{biblatex}

\addbibresource{Bibliography.bib}

\setmainfont{Georgia}

\begin{document}

\section*{Introduction}

The impetus for my research is an article by Larson outlining some of the
bizarre outcomes stemming from baby naming laws in various US states. In
particular, it is legal in New Jersey to name your child "Adolf Hitler", while
in California, it is expressely prohibited to name your child "María", that is,
with the proper Spanish accentuation. Baby naming being one of the most visceral
acts of linguistic power, I wished to invesigate further the relations between
language and government power.

\section*{Social Perspective}

Sociolinguistics is grounded in the idea that language does not exist in a
theoretical vacuum; instead, it carries social evaluations with real outcomes on
those who engage in a particular language variety. Thus we will look at how
these social evaluations and their juridical manifestations have played out over
the history of the United States, the effects of various laws on American
language communities, a more general paradigm of political linguistics, and the
situation in California more closely.

\section*{Political Linguistics}

\section*{Deficit Mindset}

\section*{Language Repression}

\section*{Language in the United States}

\section*{Bilingual Education}

\end{document}
